Despite their expressivity and robustness, interactive theorem provers (ITPs)
can be prohibitively costly to use in large-scale formalizations due to the burden of
interactively proving goals.
%
Discharging some of these goals via automatic theorem provers, such as
satisfiability modulo theories (SMT) solvers, is a known way of improving the
usability of ITPs.
%
This thesis describes a novel integration between the ITP
Lean 4 and the SMT solver cvc5.

Given the encoding of some Lean goal as an SMT problem and a proof from
cvc5 of the encoded problem, we show how to lift this proof into a proof
of the original goal.
%
This requires proving the correctness, inside Lean, of the steps taken by the
solver. Thus Lean's proof checker will accept the SMT proof as a proof of the original goal, in case this process is successful.

This set of techniques is part of the joint project Lean-SMT, which aims to create a
tactic in Lean that implements the whole pipeline, that is, from a goal in
Lean, translate it into a query in the solver's language, try to prove it using
a solver and produce a proof and, in case it is successful, lift the proof produced,
closing the original goal in Lean (which is done by the techniques presented here).
