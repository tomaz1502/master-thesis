Despite their expressivity and robustness, interactive theorem pro\-vers (ITPs)
can be quite costly to use in large-scale formalizations due to the burden of
interactively proving goals.
%
Discharging some of these goals via automatic theorem provers, such as
satisfiability modulo theories (SMT) solvers, is a known way of improving the
usability of ITPs.
%
This thesis describes a novel integration between the ITP
Lean 4 and the SMT solver cvc5.

Assuming an encoding of the Lean goal as an SMT problem and that cvc5
generates a proof for the encoded problem, we show how to lift this proof into
a proof of the original Lean goal.
%
This requires proving the correctness, inside Lean, of the steps taken by the
solver, as well as decoding the terms in the proof into the original Lean
ones. Thus Lean can accept the SMT proof as a proof of the original goal.

This tool is part of the joint project Lean-SMT, which aims to create a
tactic in Lean that implements the whole pipeline, that is, from a goal in
Lean, translate it into a query in SMT-Lib format, try to prove it using
a SMT solver and, in case it is successful, lift the proof produced,
closing the original goal in Lean (which is done by our tool).
%
All the other steps of the pipeline are in an advanced stage of development.
