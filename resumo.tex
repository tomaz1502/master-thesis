Apesar de sua expressividade e robustez, assistentes
de demonstra\c{c}\~ao podem ser proibitivamente custosos
para serem usados em formaliza\c{c}\~oes de grande escala,
dada a dificuldade de produzir as demonstra\c{c}\~oes
interativamente.
%
Atribuir a responsabilidade de demonstrar algumas das
proposi\c{c}\~oes a demonstradores autom\'aticos de teoremas,
como solucionadores de satisfatibilidade modulo teorias (SMT),
\'e um jeito reconhecido de melhorar a usabilidade de
assistentes de demonstra\c{c}\~ao.
%
Essa disserta\c{c}\~ao descreve uma nova integra\c{c}\~ao
entre o assistente de demonstra\c{c}\~ao Lean 4 e o
solucionador SMT cvc5.

Dada uma codifica\c{c}\~ao de um teorema declarado em Lean como um
problema SMT e uma demonstra\c{c}\~ao provida pelo cvc5 para
o problema codificado, n\'os mostramos como traduzir essa
demonstra\c{c}\~ao para uma que certifique o teorema original em
Lean.
%
Para isso \'e necess\'ario demonstrar a corretude, em Lean, dos
passos l\'ogicos tomados pelo solucionador. Desse modo, caso o processo
seja bem sucedido, o verificador de demonstra\c{c}\~oes de Lean aceitar\'a a
demonstra\c{c}\~ao SMT dada pelo solucionador para o teorema original.

Essas t\'ecnicas s\~ao integradas no projeto Lean-SMT,
que tem como objetivo criar uma t\'atica em Lean que implemente
o processo completo, isto \'e, a partir de um teorema em Lean,
traduzi-lo para um problema formulado na linguagem do solucionador SMT,
invocar um solucinador para tentar resolv\^e-lo e produzir
uma demonstra\c{c}\~ao, e, caso ele seja bem-sucedido,
tentar traduzi-la para certificar o teorema original em Lean (o que
\'e feito pelas t\'ecnias apresentadas aqui).
