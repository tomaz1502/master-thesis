In this chapter we show how to lift proofs produced by cvc5 into
proofs of Lean statements via certified transformations.
First, we introduce our representation of MSFOL terms inside Lean.
Then, we will show how to state and prove theorems stating relations
between terms in this representation. We will define theorems matching
the inference rules used by cvc5, in a way that the solver will be
capable of expressing its proofs inside of Lean and have their validity
checked by the kernel.


% Then, in Section~\ref{sec:certified_rcons},
% we show how to prove theorems about terms represented in this way and
% how to lift these proofs into proofs about native Lean terms.
% Lastly, in Section~\ref{sec:downsides}, we present some downsides of this
% approach.


% First, in Section~\ref{sec:gen-scripts}, we present the format used by
% cvc5 to print its proofs as Lean scripts.
% Then, in Section~\ref{sec:certified_rcons}, we outline the necessary
% steps to lift these scripts into a sequence of certified transformations,
% which will have its validity checked by Lean's kernel\footnote{Our implementation of this lifting can be found at \url{https://github.com/tomaz1502/lean-smt/tree/main/Smt/Reconstruction/Certified}}.
% Lastly, in Section~\ref{sec:downsides}, we present some downsides of this approach.
