
\subsection{Description of the Problem}

Satisfiability Modulo Theories (SMT)~\cite{smt} is a generalization of the Boolean
Satisfiability Problem (SAT). In this version, the underlying logic is First Order Logic
instead of Propositional Logic, that is, the input formula for some instance of the
problem can contain quantifiers binding variables which will affect the satisfiability of
that instance. Another addition is the inclusion of a set of theories that allows the
problem to refer to
variables of different domains. More precisely, a theory consists of a sort (for instance, integers) over which a subset of the variables of the problem can range over and a set of operations (for instance, addition and comparison operations) can be applied to those variables.

TODO:
\begin{itemize}
  \item give examples to make it more clear
  \item motivate
  \item define MSFOL like bohme?
\end{itemize}


\subsection{SMT-LIB ?}

\subsection{SMT Solvers}
