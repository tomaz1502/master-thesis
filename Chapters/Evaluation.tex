In this chapter we will present the evaluation of our tool.
%
The main focus of our experiments is to measure the coverage
of our proof reconstruction module.
%
More concretely, we want to check how often our framework, when
integrated with cvc5's proof-generation mechanisms, is capable
of reconstructing the solver's results.
%
We will investigate whether there are any mismatches between
the semantics of some inference rule and the corresponding tactic,
as well as assess cvc5's capability of completely expressing
its reasoning through proof certificates.

As we have explained before, we do not have as a goal to implement a hammer
for Lean that would be as effective as SMTCoq or Sledgehammer.
%
Considering the time constraints of our project and the fact that
Lean currently does not have any project that attacks this problem, our goal
is to determine whether it is possible to perform proof reconstruction
using cvc5's output in Lean, and how to do it.
%
We believe that this is an important first step towards having
a hammer for Lean that is fast and powerful.
%
With this in mind, we will not compare the performance of our tool
with these established hammers.

\section{Strategy}

In order to evaluate our tool, we used the benchmarks provided
by the SMT-LIB initiative~\cite{smtlib_initiative}. The problems in this benchmark
are categorized by the combination of theories they are expressed in.
Given our support for the theories of Equality and Uninterpreted Functions,
Linear Integer Arithmetic and Linear Real Arithmetic, we can reconstruct problems
from the combinations of theories identified in the benchmark by QF\_LIA, QF\_LIRA,
QF\_LRA, QF\_UF, QF\_UFLIA, QF\_UFLRA (QF stands for quantifier free and is used
to identify problems that do not have quantifiers in their definition).
%
For each problem within this set, we ran cvc5\footnote{We used the version of
  cvc5 from this commit: \url{https://github.com/HanielB/cvc5/tree/b2340f42639733a0ef9523aee2b68f0bf062a5a7}}
with a timeout of 600s to determine
which ones were unsatisfiable and generate a proof in Lean\footnote{It is necessary to pass the following
  flags to cvc5 to generate the proofs: \texttt{--dump-proofs --dag-thresh=0
    --proof-granularity=theory-rewrite --proof-format=lean}} for their unsatisfiability,
resulting in a total of 6102 proofs.
%
From our experience, proofs exceeding 1MB are too costly to be checked
with the current state of our framework, therefore, we have filtered out the proofs
that surpassed this threshold, which amounted to a total of 877 proofs.

For the purpose of checking these proofs, we have written a Lean script that, given
a Lean file, loads our framework and checks if all the proofs contained in the file
are correct\footnote{The script can
  be found at \url{https://github.com/tomaz1502/process_lean_smt/blob/main/Main.lean}}.
%
It does not check whether the theorem in Lean indeed corresponds to the original SMT query.
%
As we have explained before, if the translation from the query to the theorem was wrong,
it is very likely that the translated proof would also be wrong, so the correctness
of the proof is a strong evidence for the correctness of the translation.
%
Using the binary produced when the script is compiled, one can reconstruct the
proofs without installing Lean.
%
It just requires that all the object files from our library (and from its dependencies)
are in the same folder as the proof that is being reconstructed.
%
Notice that this script has to load all the libraries in each run, which heavily impacts
its performance.

All experiments were run on a server equipped with 32 processors Intel(R) Xeon(R)
CPU E5-2620 v4 2.10GHz and 125.79 GiB RAM, running Ubuntu 20.04,
kernel 5.4.0-132-generic, with 8GB of memory for each job.

\section{Analysis}

% We refrain from comparing the performance of our checker with other similar tools (such as
% SMTCoq and Sledgehammer) due to the difficulty of fairly evaluating
% the results. This difficulty comes from the deep differences between the ITPs'
% implementations.
% %
% Isabelle, for instance, runs on a language that uses
% garbage collection, which would heavily impact the efficiency of
% Sledgehammer.
% %
% Furthermore, as we have pointed out in the introduction, currently
% there are no other hammers for Lean, which we could potentially compare
% to our tool.
% %
% Also, there is currently no way to run SMTCoq and Sledgehammer for proof checking
% outside the corresponding ITPs, which also makes the comparison difficult.

% We evaluate our tool for proof reconstruction coverage.
% %
% We use the version of cvc5 from\footnote{\url{https://github.com/HanielB/cvc5/tree/b2340f42639733a0ef9523aee2b68f0bf062a5a7}}
% to generate Lean proofs from the
% problems in the SMT-LIB benchmark library\footnote{\url{https://smtlib.cs.uiowa.edu/benchmarks.shtml}} whose
% logic we support (which are QF\_LIA, QF\_LIRA, QF\_LRA, QF\_UF, QF\_UFLIA, QF\_UFLRA), with a 600s timeout\footnote{It is necessary to pass the following flags to cvc to generate the
% proofs: \texttt{--dump-proofs --dag-thresh=0 --proof-granularity=theory-rewrite --proof-format=lean}}.
% The cvc5 solver produced 6102
% proofs. We have only considered proofs whose size is smaller than
% 1mb, which is a total of 877 proofs. From our experience, proofs that are
% bigger than this take too much time to be elaborated and reconstructed,
% given the current state of our tool.


% \section{Analysis}

% The results we have found are presented in Table~\ref{bench}, separated
% by SMT theory. From a total of 877 proofs, our tool could reconstruct
% successfully 281 of them. The reconstruction of the other 581 proofs
% failed for the following reasons:

% \begin{itemize}
%   \item 182 failed due to a mismatch in the application of some function. These could be either an issue in cvc5's printer or an error in the implementation of the tactics.
%   \item 125 failed due some tactic exceeding Lean's resources.
%   \item 105 failed due cvc5 assuming that Lean's \texttt{Ne} operator is $n$-ary, while it is binary.
%   \item 11 failed since the tactics \texttt{arithMulPos} and \texttt{arithMulNeg} currently do not suppport mixing of integers and rations between its parameters \texttt{l} and \texttt{r}. These errors are solved if cvc5 prints these steps casting all the integers to rationals.
%   \item 1 failed due to cvc5 not printing the \texttt{factor} tactic correctly in a specific case.
%   \item 22 failed for exceeding the limit of time.
%   \item 154 failed for exceeding the limit of memory.
%   \item 1 failed for producing a stack overflow.
% \end{itemize}

% Therefore, our tool failed in 302 proofs due to performance issues, and
% possibly in some subset of the 182 described in the first bullet due to
% implementation errors. These informations indicate to us that our main
% focus should be on improving the efficiency of our tool. As we will
% show in Chapter~\ref{chap:future}, we have promising ideas to improve
% the performance of the reconstruction process.


% \begin{table}[t]
% \centering
% \begin{tabular}{ l l l l l l l }
% \toprule
% Theory & Total & Valid & Invalid & Timeout & Memout & Error \\ \midrule
% QF\_LIA & 236 & 30 & 161 & 0 & 45 & 0 \\ \midrule
% QF\_LRA & 205 & 91 & 110 & 4 & 0 & 0 \\ \midrule
% QF\_UF & 300 & 64 & 117 & 9 & 109 & 1 \\ \midrule
% QF\_UFLIA & 106 & 67 & 30 & 9 & 0 & 0  \\ \midrule
% QF\_UFLRA & 31 & 29 & 2 & 0 & 0 & 0 \\ \bottomrule
% \end{tabular}
% \caption{Total proof-checking success, fails and time per theory.}\label{bench}
% \end{table}

% %                  total  unsat  sat  timeout  memout  error  time_cpu    memory
% % benchmark config
% % QF_LRA              205     91  110        4       0      0   16998.6   98538.6
% % QF_UF               300     64  117        9     109      1   51072.6 1030329.4
% % QF_UFLIA            106     67   30        9       0      0   56169.0  167205.5
% % QF_UFLRA             31     29    2        0       0      0    1696.0    4108.3
