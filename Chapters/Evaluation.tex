In this chapter we will present the evaluation of our set of tactics.
%
The main focus of our experiments is to measure the coverage
of our proof reconstruction library.
%
More concretely, we want to check how often our framework, when
integrated with cvc5's proof-generation mechanisms, is capable
of reconstructing the solver's reasoning.
%
We will investigate whether there are any mismatches between
the semantics of some inference rule and the corresponding tactic,
as well as assess cvc5's capability of completely expressing
its reasoning through proof certificates.

As we have explained before, we do not have as a goal to implement a hammer
for Lean that would be as efficient as SMTCoq or Sledgehammer.
%
Considering the time constraints of our project and the fact that
Lean currently does not have any project that attacks this problem, our goal
is limited to determine whether it is possible to perform proof reconstruction
using cvc5's output in Lean, and how to do it.
%
We believe that this is an important first step towards having
a hammer for Lean that is fast and powerful.
%
With this in mind, we will not compare the performance of our tool
with these established hammers.

\section{Strategy}

In order to evaluate our tool, we used the benchmark provided
by the SMT-LIB initiative~\cite{smtlib_initiative}. The problems in this benchmark
are categorized by the combination of theories they are expressed in.
Given our support for the theories of Equality and Uninterpreted Functions,
Linear Integer Arithmetic and Linear Real Arithmetic, we can reconstruct problems
from the combinations of theories identified in the benchmark by QF\_LIA, QF\_LIRA,
QF\_LRA, QF\_UF, QF\_UFLIA, QF\_UFLRA (QF stands for quantifier free and is used
to identify problems that do not have quantifiers in their definition).
%
Ideally, we would be able to reconstruct any proof for a query
in this set of theories.
%
SMTCoq and Sledgehammer would be able to reconstruct proofs in the
same theories and in the theory of Bitvectors.
%
For each problem within this set, we executed a version of cvc5 that has the experimental feature of printing its proofs as Lean scripts\footnote{We used the version of
  cvc5 from this commit: \url{https://github.com/HanielB/cvc5/tree/b2340f42639733a0ef9523aee2b68f0bf062a5a7}}.
We gave it a timeout of 600s for each problem to determine
which ones were unsatisfiable and generate a proof\footnote{It is necessary to pass the following
  flags to cvc5 to generate the proofs as Lean scripts: \texttt{--dump-proofs --dag-thresh=0
    --proof-granularity=theory-rewrite --proof-format=lean}} for their unsatisfiability,
resulting in a total of 6102 proofs.
%
From our experience, proofs exceeding 1MB are too costly to be checked
with the current state of our framework, therefore, we have filtered out the proofs
that surpassed this threshold, which amounted to a total of 876 proofs.

For the purpose of checking these proofs, we have written a Lean script that, given
a Lean file, loads our framework and checks if all the proofs contained in the file
are correct\footnote{The script can
  be found at \url{https://github.com/tomaz1502/process_lean_smt/blob/main/Main.lean}}.
%
It does not check whether the theorem in Lean indeed corresponds to the original SMT query.
%
As we have explained before, if the translation from the query to the theorem was wrong,
it is very likely that the translated proof would also be wrong, so the correctness
of the proof is a strong evidence for the correctness of the translation.

Using the binary produced when our script is compiled, one can reconstruct the
proofs without installing Lean.
%
It just requires that all the object files from our library (and from its dependencies)
are in the same folder as the binary.
%
Notice that this script has to load all the libraries in each run, which heavily impacts
its performance.
%
We emphasize again that our goal with this evaluation is to test the coverage of our
library and not its performance.
%
Writing a tool for proof reconstruction that prioritizes performance requires
giving more attention to different aspects of the system and is another
project in itself.
%
All experiments were run on a server equipped with 32 processors Intel(R) Xeon(R)
CPU E5-2620 v4 2.10GHz and 125.79 GiB RAM, running Ubuntu 20.04,
kernel 5.4.0-132-generic, with 8GB of memory for each job.

\section{Analysis}

The results we obtained are presented in Table~\ref{results_rcons}, categorized
by theory combination. From a total of 876 proofs, our tool could reconstruct
successfully 281 of them. The reconstruction of the other 596 proofs
failed for the following reasons (note that some errors occurred in more than one test case):

\begin{table}[t]
\centering
\begin{tabular}{ c c c c c c c }
\toprule
Theory  & Total   & Succeeded & Error while checking & Timeout & Memout \\ \midrule
QF\_UF    & 299   & 64    & 117     & 9       & 109  \\ \midrule
QF\_LIA   & 236   & 30    & 161     & 0       & 45   \\ \midrule
QF\_LRA   & 205   & 91    & 110     & 4       & 0    \\ \midrule
QF\_UFLIA & 106   & 67    & 30      & 9       & 0    \\ \midrule
QF\_UFLRA & 31    & 29    & 2       & 0       & 0    \\ \bottomrule
\end{tabular}
\caption{Result of proof checking per theory.}\label{results_rcons}
\end{table}

\begin{enumerate}
  \item 89 failed due to steps asserting equalities between terms of different types, which is allowed in cvc5's calculus but is not allowed in Lean's type system.
  \item 109 failed due to cvc5 assuming that Lean's \texttt{Ne} operator is $n$-ary, while it is binary.
  \item 5 failed because cvc5 was implicitly reordering the operands of an addition or a disjunction.
  \item 22 failed for exceeding the limit of time.
  \item 154 failed for exceeding the limit of memory.
  \item 125 failed for exceeding the limit of time or memory stipulated by Lean for the execution of a single tactic.
  \item 11 failed since the tactics \texttt{arithMulPos} and \texttt{arithMulNeg} did not suppport mixing of integers and rations between its parameters \texttt{l} and \texttt{r}.
  \item 28 failed because Lean incorrectly inferred the type of some term in an intermediate congruence step.
  \item 75 failed since Lean can't automatically lift an equality between integers to an equality between rationals, which is implicitly done by cvc5.
\end{enumerate}

After running the benchmark, we fixed issue 7 by adapting the tactic to perform the
appropriate castings if the terms passed as arguments had different types. We also
fixed issues 8 and 9 by replacing the congruence theorem with a tactic that uses
Lean's rewriter to simulate the congruence rule.
This change has two advantages: first, since the rewrite is
capable of coercing the types when needed, we can skip the process
of matching the types of the hypothesis with the types expected by the congruence theorem, which will
eliminate the need for type inferences that were previously done incorrectly; second, Lean's rewriter
can automatically lift an equality between integers into an equality between rationals if needed, solving
the last problem. This is needed, for instance, if we have an hypothesis \mintinline{lean}{(0 : Int) = b}
and cvc5 wants to use it to prove \mintinline{lean}{a + b = a + 0}, where \texttt{a}
is a rational. All these fixes were tested and are working properly.

Issue 1 should be fixed inside cvc5's proof printing module, by adding the appropriate type coercions
each time an ill-typed equality is built.
%
Issue 2 can be fixed either in Lean, through the introduction of an n-ary inequality operator,
or in cvc5, by transforming the n-ary inequality into a conjunction of multiple inequalities.
%
Issue 3 should also be fixed inside cvc5. It is necessary to add intermediary steps to reorder the
additions and disjunctions before using them. Disjunctions can be reordered using our \texttt{permutateClause}
tactic and additions can be reordered using mathlib's \texttt{linarith} tactic.

The results of the benchmark point out that, despite its performance issues, our tool can
cover most of the reasoning performed by cvc5. Within the issues indicated by our set of
test cases, only the first three remain as covering problems, and they should be solved
by adapting cvc5's printer. This outcome make clear that the main focus of the future work
should be on optimizing the performance of the tool. In Chapter~\ref{chap:future}, we
will show some promising ideas we have for improving it.


% [2]: And.intro uses a variable introduced by scope with a wrong type
% [3]: Eq Int Rat
% [4]: casting problem (congrHAdd + binrel) (FIXED)
% [5]: cvc5 assuming implicitly hAdd to be commutative?
% [6]: congr expected equality between rat, got int
% [7]: congrHAdd: premise has more subexpressions casted to rat then the goal specified by cvc5 (FIXED)
% [8]: congrIte / iteIntro missing universe parameter (FIXED)
% [9]: Resolution permutating props

% Therefore, our tool failed in 302 proofs due to performance issues, and
% possibly in some subset of the 182 described in the first bullet due to
% implementation errors. These informations indicate to us that our main
% focus should be on improving the efficiency of our tool. As we will
% show in Chapter~\ref{chap:future}, we have promising ideas to improve
% the performance of the reconstruction process.


