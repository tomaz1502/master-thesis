We refrain from comparing its performance to other similar tools (such as
SMTCoq and Sledgehammer) due to the difficulty of fairly evaluating
the results. This difficulty comes from the deep differences between the ITPs'
implementations. Isabelle, for instance, runs on a language that use
garbage collection, which would heavily impact the efficiency of
sledgehammer. As we have pointed out in the introduction, currently
there are no other hammers for Lean, which we could potentially compare
to our tool.

We evaluate our tool for proof reconstruction coverage.
We use the commit
\tom{insert commit here} of cvc5 to generating Lean proofs from the
problems in the SMT-Lib benchmark library\footnote{\url{https://smtlib.cs.uiowa.edu/benchmarks.shtml}} whose
logic we support, with a 1800s timeout. The cvc5 solver produced 6102
proofs. We have only considered proofs whose size is smaller than
1mb, which is a total of 877 proofs. From our experience, proofs that are
bigger than this take too much time to be elaborated and reconstructed,
given the current state of our tool.
All experiments were run on a server
equipped with 16 processors Intel(R) Xeon(R) CPU E5-2637 v4 3.50GHz and
62.79 GiB RAM, running Ubuntu 20.04, kernel 5.4.0-132-generic, with 8GB of memory for each job.

We have written a Lean script for reconstructing the proofs\footnote{\url{https://github.com/tomaz1502/process_lean_smt/blob/main/Main.lean}}. Using the binary that is produced when the script is compiled, one can reconstruct the proofs without installing Lean.
It just requires that all the object files from our library (and from its dependencies) are in the same folder as the proof that is being reconstructed.
Our experiments were performed using this script.

\section{Analysis}

The results we have found are presented in Table~\ref{bench}, separated
by SMT theory. From a total of 877 proofs, our tool could reconstruct
successfully 281 of them. The reconstruction of the other 581 proofs
failed for the following reasons:

\begin{itemize}
  \item 182 failed due to a mismatch in the application of some function. These could be either an issue in cvc5's printer or an error in the implementation of the tactics.
  \item 125 failed due some tactic exceeding Lean's resources.
  \item 105 failed due cvc5 assuming that Lean's \texttt{Ne} operator is $n$-ary, while it is binary.
  \item 11 failed since the tactics \texttt{arithMulPos} and \texttt{arithMulNeg} currently do not suppport mixing of integers and rations between its parameters \texttt{l} and \texttt{r}. These errors are solved if cvc5 prints these steps casting all the integers to rationals.
  \item 1 failed due to cvc5 not printing the \texttt{factor} tactic correctly in a specific case.
  \item 22 failed for exceeding the limit of time.
  \item 154 failed for exceeding the limit of memory.
  \item 1 failed for producing a stack overflow.
\end{itemize}

Therefore, our tool failed in 302 proofs due to performance issues, and
possibly in some subset of the 182 described in the first bullet due to
implementation errors. These informations indicate to us that our main
focus should be on improving the efficiency of our tool. As we will
show in Chapter~\ref{chap:future}, we have promising ideas to improve
the performance of the reconstruction process.


\begin{table}[t]
% \centering
\begin{tabular}{ l l l l l l l l l }
\toprule
Theory & Total & Valid & Invalid & Timeout & Memout & Error  & Time &  Memory \\ \midrule
QF\_LIA & 236 & 30 & 161 & 0 & 45 & 0 & 68088.8 & 1262055.6 \\ \midrule
QF\_LRA & 205 & 91 & 110 & 4 & 0 & 0 & 16998.6 & 98538.6 \\ \midrule
QF\_UF & 300 & 64 & 117 & 9 & 109 & 1 & 51072.6 & 1030329.4 \\ \midrule
QF\_UFLIA & 106 & 67 & 30 & 9 & 0 & 0 & 56169.0 & 167205.5 \\ \midrule
QF\_UFLRA & 31 & 29 & 2 & 0 & 0 & 0 & 1696.0 & 4108.3 \\ \bottomrule
\end{tabular}
\caption{Total proof-checking success, fails and time per theory.}\label{bench}
\end{table}



\tom{what is the unit in time\_cpu and memory?} \\
\tom{the numbers on the items sums to 601, but they should sum to 586}

%                  total  unsat  sat  timeout  memout  error  time_cpu    memory
% benchmark config
% QF_LRA              205     91  110        4       0      0   16998.6   98538.6
% QF_UF               300     64  117        9     109      1   51072.6 1030329.4
% QF_UFLIA            106     67   30        9       0      0   56169.0  167205.5
% QF_UFLRA             31     29    2        0       0      0    1696.0    4108.3
