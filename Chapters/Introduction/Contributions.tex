We aim to build a system that takes proofs of the unsatisfiability of
SMT queries produced by the SMT solver cvc5~\cite{cvc5} and reconstructs them as proofs in the ITP Lean 4~\cite{lean}.
%
For this purpose, we present a set of tactics and theorems essential
to the integration between Lean and cvc5.
%
The main motivation of this project is that despite the fact that Lean is
emerging as a promising programming language and ITP and being
widely used by mathematicians in large-scale
formalizations~\cite{mathlib, scholze}, there is currently no way to
interact with SMT solvers from it, even though these systems have been
central in previous developments of proof automation in ITPs, as we will show in Sections~\ref{sec:smtcoq}
and~\ref{sec:sledgehammer}. The contribution of the present work
is essential to develop this kind of project using Lean.


We use the cvc5 solver because it already has a module for exporting proofs as
Lean scripts~\cite{Barbosa2022}, using a representation of the SMT terms\footnote{For more details
about the SMT term language, see SMT-Lib~\cite{smtlib}.} as Lean's native terms.
Each proof produced by cvc5 consists of a set of logical inferences starting from the
hypothesis until the goal is inferred. In order to reconstruct such proofs inside of
Lean, it is necessary to prove the correctness of these inference rules inside the
ITP.\ Our main contribution is to provide a set of tactics and theorems matching the set
of rules used by the ATP, enabling the verification of the rules through Lean's kernel.
Furthermore, these tactics and theorems operate over native Lean terms, facilitating the process of using this module as part of a system that proves theorems originally stated
in Lean's language through proofs produced by cvc5.

Note that the goal of the present work is to reconstruct in Lean proofs produced by
cvc5 for SMT problems. In order to have a full integration between Lean and cvc5 we also
need a module for translating Lean goals into an equivalent SMT problem.
However, our project is being used
as part of the joint project Lean-SMT\footnote{The code for the project can be found at \url{https://github.com/ufmg-smite/lean-smt}}, that aims to implement a tactic in Lean
that would perform the complete process, that is, starting from a Lean goal, translating it to an SMT query, invoking the solver to try to prove it and lifting the proof produced (in case it is found) to Lean's language, so that it can be used as a proof for the original goal.
