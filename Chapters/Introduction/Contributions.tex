We have built a system that takes proofs of the unsatisfiability of
SMT queries produced by the SMT solver cvc5~\cite{cvc5} and
reconstructs them as proofs in the ITP Lean 4~\cite{lean},
which has as its main goal the capacity of reconstructing the largest possible
range of proofs produced by cvc5.
%
We present in detail two strategies well-known in the literature for doing this translation.
%
We have effectively applied one of them, and we share our insights
regarding the reasons we believe it is more appropriate than the other
for our goals.

The main motivation of this project is that, despite the fact that Lean is
emerging as a promising programming language and ITP and being
widely used by mathematicians in large-scale
formalizations~\cite{scholze, mathlib}, there is currently no way to
interact with SMT solvers from it, even though these systems have been
central in previous developments of proof automation in ITPs, as we will show in Section~\ref{sec:related}. The contribution of the present work
is essential to develop this kind of project using Lean.

The cvc5 solver has a module for post-processing its proofs,
translating and printing them in a format recognized by Lean~\cite{Barbosa2022} (which is the reason why we chose it).
Each proof produced by cvc5 consists of a set of logical inferences starting from the
hypothesis until the goal is inferred. In order to reconstruct such proofs inside of
Lean, it is necessary to prove the correctness of these inference rules inside the
ITP.\
Our main contribution is to provide a set of tactics and theorems matching the set
of rules used by the ATP, enabling the verification of the translated proofs through Lean's kernel.
Furthermore, it is possible to leverage these proofs to discharge the
responsibility of proving theorems originally stated in Lean to cvc5, which is
the main use case for our module. There are other modules that need to be implemented
in order to fully automate this process (which we will describe in Section~\ref{sec:related}), that are out of the scope of the present project.
However, our project is being used as part of the joint project
LeanSMT\footnote{The code for the project can be found at \url{https://github.com/ufmg-smite/lean-smt}}, that aims to implement a tactic in Lean
that would perform this process of discharging proofs. Starting from a Lean goal, the tactic would translate
it to an SMT query, then invoke the solver to try to prove it and lift the proof produced
(in case it is found) to Lean's language, so that it can be used as a proof for the original
goal.

Another contribution of our work is the first checker for cvc5 proofs
(which are expressed in the theories we support) verified by an established ITP.\@
%
Other checkers exist (\cite{carcara} and~\cite{lfsc}, for instance), but one has to
trust their source code to verify proofs with them. While it is necessary
to trust Lean's kernel to verify proofs with our module, Lean is a tool
with thousands of users constantly writing proofs in it. Therefore,
if it had an inconsistency in its kernel, it is very likely that
an user would have already discovered it.
%
While this is an useful feature of our tool, it is important to highlight that
we did not design it as a proof checker and it lacks optimization for this purpose,
resulting in a suboptimal performance for this task.

% checking proofs is not its main use case and the tool was not optimized for this.
