Given this context, we present a tool that would be an essential
part of the integration between the ITP Lean 4~\cite{lean} and the SMT solver cvc5~\cite{cvc5}.
%
Specifically, we aim to build a system that takes proofs of the unsatisfiability of
SMT queries produced by cvc5 and reconstructs and checks them using Lean.
%
The main motivation of this project is that despite the fact that Lean is
emerging as a promising programming language and proof assistant and being
widely used by mathematicians in large-scale
formalizations~\cite{mathlib, scholze}, there is currently no way to
interact with SMT solvers from it, even though these systems have been
central in previous developments of proof automation in ITPs, as seen in Sections~\ref{sec:smtcoq}
and~\ref{sec:sledgehammer}. The contribution of the present work
would enable a faster development of this kind of project using Lean.

We use the cvc5 solver because it already has a module for exporting proofs as
Lean scripts~\cite{Barbosa2022}, using a representation of the SMT terms\footnote{For more details
about the SMT term language, see SMT-LIB~\cite{smtlib}.} as an inductive type in Lean.

Note that, as opposed to SMTCoq and Sledgehammer that implements all the three modules
of a hammer (as described in Section~\ref{sec:hammering}), our system only implements
the third module for now, despite the end goal of this project being to implement the
complete integration. Also, the reconstruction technique that we use is the first one
listed in Section~\ref{sec:hammering}, as we will describe in the next sections.
