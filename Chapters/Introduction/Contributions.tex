Given this context, we present a set of tools that would be an essential
part of the integration between the ITP Lean 4~\cite{lean} and the SMT solver cvc5~\cite{cvc5}.
%
Specifically, we aim to build a system that takes proofs of the unsatisfiability of
SMT queries produced by cvc5 and reconstructs and checks them using Lean.
%
The main motivation of this project is that despite the fact that Lean is
emerging as a promising programming language and proof assistant and being
widely used by mathematicians in large-scale
formalizations~\cite{mathlib, scholze}, there is currently no way to
interact with SMT solvers from it, even though these systems have been
central in previous developments of proof automation in ITPs, as we will show in Sections~\ref{sec:smtcoq}
and~\ref{sec:sledgehammer}. The contribution of the present work
would enable a faster development of this kind of project using Lean.

We use the cvc5 solver because it already has a module for exporting proofs as
Lean scripts~\cite{Barbosa2022}, using a representation of the SMT terms\footnote{For more details
about the SMT term language, see SMT-LIB~\cite{smtlib}.} as an inductive type in Lean.
However, these proofs are not fully verified by Lean's checker. Instead, a set of
axioms are declared in Lean, representing all the logical rules that cvc5 uses to prove
theorems, and the ITP only checks whether the rules were applied correctly and whether
the end result of applying all the rules in the proof is, indeed, the required one.
Our main contributions are to eliminate the need of increasing the trusted base by introducing
axioms and to make the proofs operate over native Lean terms, as opposed to terms
of the inductive type that represents SMT terms.

Note that we do not intend to implement a tool that performs the full integration
between Lean and cvc5. For instance, we do not implement a module for translating
Lean goals into an equivalent SMT problem. However, our set of tools is being used
as part of the project Lean-SMT, that
