A mechanized proof is a proof written in a language recognized by a computer, so that its validity can be checked by a verifier. An important application of these artifacts is formalizing mathematical theories. Indeed, there are well-known examples of successful formalizations. One of them is the mechanization of the proof of a theorem regarding Perfectoid Spaces~\cite{scholze}, done by the fields-medalist mathematician Peter Scholze together with the community of a system called Lean~\cite{lean}. Scholze proved the theorem using pen and paper, but was unsure of the result due to its complexity. With the help of Lean's community, he could translate the theorem and the proof to the language of Lean. Once the rewrite was complete, the system could find errors in the proof, and, after addressing all of them and obtaining confirmation from Lean's checker that the proof was correct, he could be sure of the correctness of the proof.

Another application of mechanized proofs is verifying the correctness of mission-critical software. Given a specification of the behavior of some program, the program is said to be correct if it respects its specification for any given input. For instance, one could specify that a sorting routine must always produce the sorted permutation of its input list. In this case, a given sorting routine is said to be correct if it indeed produces the desired permutation, for every list it receives. There are a variety of techniques to obtain correctness evidence for a software. The most common one is the development of tests. Tests are easy to be written and effective to find errors in programs. In fact, this approach is enough for a large amount of problems that are solved by software engineering. However, tests cannot guarantee that a program does not have flaws in general, since the number of valid inputs is almost always exceedingly large, or infinite. This kind of guarantee is extremely important for mission-critical software, that is, systems that have critical responsibilities, such as the control of airplanes or medical equipment. In this context, one promising alternative is to use a mechanized proof of the correctness of the software as an evidence for its safety.

The process of generating mechanized proofs can be divided
into two categories: interactive and automatic.
Interactive theorem provers (ITPs) allow users to define
conjectures, then attempt to manually write proofs for them,
relying on the tool to organize the set of hypothesis and
how the proof obligations changed step-wisely through the
proof, as well as to ensure the correctness of each step
according to a trusted kernel.
%
In order to keep the kernel simple and small (and, therefore,
easy to be trusted), their implementation usually just
straightforwardly checks the logic rules from the logic
implemented by the ITP.\ Because of this, each step must be
explicitly stated by the user, making the tool costly to be
used. However, some ITPs offer the possibility to facilitate
the process of writing proofs through the usage of tactics.
A tactic is a procedure that can potentially inspect the
set of hypothesis and the proof obligations and manipulate
them. Usually they are used to simulate common proof
techniques, such as case analyses and induction. The
validity of each tactic call is verified by the kernel
of the ITP to ensure that it does not compromise the tool's
soundness.

Automatic theorem provers (ATPs), on the other hand,
only require the user to define a conjecture, proceeding
automatically to determine whether there exists a proof
for it, or possibly providing a counter-example otherwise.
%
There are a variety of tools that use different
techniques to automatically reason about logical propositions.
We will focus on Satisfiability Modulo Theory~\cite[ch. 33]{handbook} solvers, systems
that determine the satisfiability of formulas written in an
extended version of First-Order Logic,
capable of incorporating structures from external domains such as arithmetic,
arrays and others. SMT solvers employ a combination of a SAT solver and
domain-specific methods to solve such formulas.
%
Although they are easier to use, ATPs require a large
codebase to implement all the algorithms necessary to execute
the search for a proof, making them more susceptible to
errors and harder to be trusted. One possible way to overcome
this trust issue is to produce a mechanized proof verifying
the correctness of the ATP.\ However, besides being a very
complex task, once the proof is done, further developments of
the ATP become harder, since the changed system has to be
verified again.

Another approach to increase the confidence in ATPs is to have them provide a
proof certificate (i.e.\ a mechanized proof) to support their results, so that it can be independently verified whether
it indeed proves the theorem in question. This approach has the downside of creating a need
for checking the proof certificate for each theorem proved.
On the other hand, as long as the proof format does not change, the implementation
of the solver can be modified without requiring a modification in the checkers. Also,
verifying the correctness of proof certificates is often much simpler than to verify
the tool itself.

Another important advantage of the second approach is that it allows the ITPs to leverage the automatic proving performed by the ATPs via the proof certificates they produce, since the requirement for accepting a proof, i.e.\ that each step is correct to its internal logic, can be applied to the ATP proof.
%
By connecting these systems, it is possible for the user of the ITP to focus on more complex steps of the proof, such as defining an induction hypothesis, while delegating the burden of other long but straightforward steps to the ATP.\
Indeed, there are many examples of projects that successfully
implement this sort of connection between ATPs and ITPs.

The paper Hammering Towards QED~\cite{hammering} describes
the most relevant ones and also outlines
all the efforts that were already made in order to integrate
interactive and automatic theorem provers. The authors employ
this information to provide a detailed description of each
component that must be implemented in a system establishing
a connection between ATPs and ITPs. Furthermore, this work
presents a series of benchmarks showing the potential of these
tools. These benchmarks tipically involve selecting a set
of theorems proved in the ITP's library and checking how often the
tool can find a proof for them without human interaction.
For instance, an experiment with the Isabelle/HOL ITP~\cite{Nipkow2002}, where
1240 theorems were sent to be proved by the corresponding tool,
resulted in a success rate of 48\%.
Those theorems are part of large-scale formalizations of complex algorithms and important mathematical facts,
such as the Fast Fourier Transform~\cite{fft} and the Fundamental Theorem of Algebra.
Some experiments demonstrated that there are theorems for which
the ATP can even find a proof shorter than those found in the ITP's library.
