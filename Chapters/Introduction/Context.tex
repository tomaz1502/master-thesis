The process of generating mechanized proofs, for example for the
correctness of a given program according to a specification, can be divided into
two categories: interactive and automatic.

Interactive theorem provers (ITPs) are mainly represented by proof assistants, in which, after defining
a theorem, the user attempts to manually write a proof for it,
relying on the tool to organize the set of hypothesis and
how the goal changed step-wisely through the proof, as well as to ensure the
correctness of each step according to a small, trusted kernel.
%
Each logic step must be explicitly stated by the user, which makes the tool
costly to be used.

Automatic theorem provers (ATPs), on the other hand,
only require the user to define a conjecture, proceeding automatically to
determine whether there exists a proof for it, or possibly providing a
counter-example if there is one.
%
Although they are easier to use, ATPs require a large
codebase to implement all the algorithms necessary to execute the search for a proof,
making them more susceptible to errors and harder to be trusted, since the
larger is the codebase, the more susceptible it is to bugs and the more complicated it is
to verify it. Besides that, once it is verified, it's development becomes freezed
(otherwise it would have to be verified again).

A common approach to address the trust issue for ATPs is to have them provide a
proof to support their results, so that it can be independently verified whether
it indeed proves the theorem in question.
%
Via these proofs the automatic proving performed by ATPs can be
leveraged by ITPs, since their requirement to accepting a proof, i.e.\ that each
step is correct according to its internal logic, can be applied to the ATP
proof.
%
By connecting these systems, the user could have all the freedom to use its own
creativity and expertise in writting proofs that the ITPs offer, while delegating
the burden of proofs that are long and monotonous to ATPs. Indeed, this
connection is so important that there are projects like Hammering Towards QED~\cite{hammering}
that outline all the efforts that were already made in order to integrate
interactive and automatic theorem provers.

