\label{sec:hammering}
As previously mentioned, Hammering Towards QED is a project
that aims to describe all the tools, which the paper calls ``hammers'',
that were created with the purpose of connecting automatic and interactive
theorem provers. Furthermore, this document also outlines
the main components that such tools usually have:

\begin{itemize}
  \item The premise selection module: it identifies
        a subset of the facts previously known by the
        ITP that are likely to be useful for the ATP to
        prove the given goals.\@
  \item The translation module: it uses the premisses selected and the original
        theorem from the ITP to formulate a query in the language of the ATP.\ This query
        must be equivalent to the original theorem, with all premisses selected
        added as hypothesis.
  \item The proof reconstruction module: it lifts the proof produced
        by the ATP into a proof that is accepted by the ITP.\@
\end{itemize}

In this dissertation we describe how we implemented a proof reconstruction module
for Lean users from cvc5 proofs.
The three main strategies used to reconstruct the proof produced
by the automatic system inside the interactive described in the paper are as follows:

The first one is known as the \textit{certified} approach~\cite{snipe}. In this case,
the hammer defines a datatype to represent
terms in the language of the ATP and a set of functions to manipulate
values of those datatypes, representing
the logical axioms that the solver uses to reason about those terms. Then, a lifting function is defined, that is,
a function that takes a value of this datatype and outputs an equivalent term in the native
language of the ITP.\ Finally, the correctness of each function is
verified with respect to the lifting function, in the sense that, if the input term
was lifted to a value that is true in the ITP's logic, then the output term will
also be true. The ATP's proof will be represented as a sequence of applications those
functions, and their correctness are proved \textit{a priori}. When checking
a specific proof, the only step that the hammer must perform is to compute the result
of the application of all the functions in the solver's output, and to check if
the final term matches with the expected one.

The second approach known as the \textit{certifying} approach~\cite{snipe}. It consists of matching
each axiom in the ATP's logic into tactics
defined in the ITP that operate directly over native terms of the system and parsing
the proof produced by the automatic solver into a sequence of applications
of those tactics, which are replayed inside the ITP.\
In this case, the proof steps are reconstructed and checked in the ITP on the
fly each time the hammer is invoked. Since there is no theorem stating the
correctness of them, it is possible that this process fails.
On the other hand,
this technique skips computations done over embedded terms, which have to be done by the
certified approach, having the potential to have a better
performance. It also diminishes the complexity of the tool, as there is no need to
define an intermediate representation inside the ITP.\

The third one is similar to the certifying approach, the only difference being that
the proof, after parsed into a sequence of applications of tactics, is stored in a separate
file instead of replayed by the ITP.\ After this, it's possible to run external tools
that perform postprocessing over this proof to simplify it~\cite{hammer_20, hammer_21}.
In some cases it's even possible to ignore a large
portion of the original proof. The proof is then replayed in the ITP, in the same way as the process
employed by the certifying approach. This technique can be inconvenient for very large proofs,
as it requires the script to be stored in the filesystem. However,
it has the advantage over the two previous methods of only requiring access to the ATP
on the first time that the proof is checked.

In this project we will be using the second approach. We give more details about this
decision in the later chapters. The next two sections describe examples of hammers.
