As previously mentioned, Hammering Towards QED is a project
that aims to describe all the tools, which the paper calls ``hammers'',
that were created with the purpose of connecting automatic and interactive
theorem provers. Besides that, this document also outlines
the main components that such tools usually have. They are
the following:

\begin{itemize}
  \item The premiss selector, that is a module that identifies
        a subset of the facts previously demonstrated in the
        ITP that are more likely to be useful in order to
        prove the given goal, to be dispatched to the ATP.\@
  \item The translation module, that builds a problem in the language of the
        ATP that corresponds to the original goal from the ITP and using
        the premisses that were selected.\@
  \item The proof reconstruction module, that lifts the proof produced
        by the ATP into a proof that is accepted by the ITP.\@
\end{itemize}

Moreover, the main strategies used to reconstruct the proof produced
by the automatic system inside the interactive one are also reported.
We give a brief description of them:

\begin{itemize}
  \item Parsing each step of the proof into predefined lemmas or tactics
        from the ITP and replay them inside the system.
  \item Use the ITP to verify a deeply embedded version of the proof received,
        and, in case it is succesful, reflect this proof inside it's checker
        to prove the original goal.
  \item Compile the proof into the ITP's source code. This implies generating
        an actual script in the native language of the interactive system
        that corresponds to the proof received. This method, as opposed
        to the previous two, has the advantage of not requiring access to
        the ATP every time the proof is checked, but only on the first time.

\end{itemize}


