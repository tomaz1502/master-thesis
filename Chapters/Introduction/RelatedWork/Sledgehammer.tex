The ITP Isabelle/HOL~\cite{Nipkow2002} has a similar tool,
namely, Sledgehammer~\cite{sledgehammer}. This system have multiple
strategies for building a proof for a theorem using ATPs. One of them
is to invoke several ATPs in parallel with the given goal as their query
and parse their output into a sequence of applications of Isabelle's
predefined lemmas and tactics that could, in thesis, prove the original
goal. This falls into the second category of strategies for proof
reconstruction presented in Section~\ref{sec:hammering}.

Another technique used by Sledgehammer is to use the proof found
by the ATP's only to scan which lemmas were used, and then discard
it and build a whole new proof, feeding the lemmas found into existing
plugins for Isabelle that already perform automatic theorem proving for
the ITP.\ In this case, Sledgehammer works as a premiss selector
(described in Section~\ref{sec:hammering}) for the other plugins.
