The ITP Isabelle/HOL~\cite{Nipkow2002} has a similar tool,
called, Sledgehammer~\cite{sledgehammer}. This system has multiple
strategies for building a proof for a theorem using ATPs. One of them
is to invoke several ATPs in parallel on the given goal
and parse their output into a sequence of applications of Isabelle's
predefined lemmas and tactics that could possibly prove the original
goal. This falls into the second category of strategies for proof
reconstruction presented in Section~\ref{sec:hammering}.

Another technique used by Sledgehammer is to use the proof found
by the ATP's only to scan which lemmas were used, and then discard
it and build a whole new proof, feeding the lemmas found into existing
plugins for Isabelle that already perform automatic theorem proving for
the ITP.\ In this case, Sledgehammer works as a premise selector
(as described in Section~\ref{sec:hammering}) for the other plugins.
