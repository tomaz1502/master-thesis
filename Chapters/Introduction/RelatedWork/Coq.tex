SMTCoq~\cite{smtcoq} is a hammer for the Coq proof assistant~\cite{Bertot2004}.
It offers tactics to prove theorems in Coq via the external proof-producing SMT
solvers veriT~\cite{Bouton2009} and CVC4~\cite{Barrett2011}. The tool can support multiple proof formats
due to a preprocessor written in OCaml, that is able to turn them
into a unified certificate in the Coq language,
which will be used as input for the plugin. Our tool explores one specific
proof format of cvc5, therefore, for now, it only supports this solver and is
less flexible in this aspect than SMTCoq.

The core of the hammer directly follows the certified transformations approach
(although recently, it was extended with new features, implemented using
certifying transformations~\cite{snipe}).
%
It has a set of certified functions representing the transformations
that are sound in three of the main domains that are available in SMT:\
Linear Arithmetic, Uninterpreted Functions and Bitvectors. The first one is used to
represent formulas involving numeric variables, the second one is
used to represent congruence relatons over variables and uninterpreted functions and
the third one is used to simulate operations over numbers as they are represented
by computers, preserving the semantics of this representation.
%
Our tool currently supports Linear Arithmetic and Uninterpreted Functions.

The way that SMT solvers prove
that a proposition is true is by showing that the negation of the proposition is unsatisfiable,
that is, false for any assignment of its free variables. This kind of proof
is parsed in SMTCoq into a sequence of applications of certified functions, which have
to transform the term representing the negation of the original goal into a
term that will be lifted to the \textit{False} proposition in Coq. Once the
ITP verifies that the sequence of steps produced by the solver indeed
produces \textit{False}, the hammer apply its main theorem, which states that if
this process was successful, then the original goal is true.

SMTCoq also differs from our work as it implements a translation module, but
it does not implement a premise selection module.
