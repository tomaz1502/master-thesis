One notable example of such integrations is SMTCoq~\cite{smtcoq}.
It is a plugin for the proof assistant Coq~\cite{Bertot2004} that
can be used as a tactic to prove theorems via their encoding into
SMT and by lifting proofs produced by the SMT solvers veriT~\cite{Bouton2009}
and CVC4~\cite{Barrett2011}. The tool relies on a preprocessor written in OCaml
to transform proof witnesses coming from different solvers into certificates in
the Coq language. The system has a set of checkers for each theory in SMT, each
one of them consisting of theorems asserting the validity of certain transformations
in the SMT terms.

% more details to be understandable
% be more high level. 'clause' has not been defined
% or for a term to be false and so on
% explain certified transformations
%  (maybe not necessary as we are citing hammering first)

All those checkers are connected by the main checker, that
is essentially a theorem stating that if all the transformations resulted in an
empty clause, then the lifting of the original term is false, for any instantiation
of its free variables. This kind of reasoning is known as proof by computational
reflection~\cite{reflection} which is an instance of Certified Transformations, which will be described
in Section~\ref{sec:certifiedVsCertifying}.
