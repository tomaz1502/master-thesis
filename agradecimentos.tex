Em primeiro lugar, eu agradeço ao Haniel, que, além de ser meu orientador e tornar tudo isso possível, se tornou um valioso amigo durante o mestrado. Haniel teve uma gentileza e eficiência excepcional para me guiar no processo. Eu devo muito a ele.

Eu agradeço também aos membros da banca, Fernando e Jeremy. Os dois são cientistas incríveis que tiveram a paciência de ler e entender a dissertação e prover uma perspectiva muito relevante sobre ela.

Nos últimos dois anos eu tive a oportunidade de conhecer e trabalhar com muitas pessoas interessantes. Em particular: Leonardo de Moura, Bruno Andreotti, Arjun Viswanathan, Hanna Lachnitt e todas as pessoas do time do cvc5. Sou grato pelo impacto positivo que tiveram em mim.

Eu fui introduzido à teoria dos tipos e verificação formal pelo Maurício Collares, um ex-professor da UFMG. Maurício é uma pessoa fantástica, com uma inteligência e gentileza fora do comum. Me sinto privilegiado por tê-lo conhecido e por todas as reuniões aprendendo Agda com ele.

Agradeço também aos meus pais, Fábio e Telma, por me criarem e pelo amor e suporte incondicional. As circunstâncias da vida têm tornado difícil o nosso encontro. Agradeço pela resiliência e compreensão deles para passar por esse desafio.

Ao longo da vida eu tive a sorte de criar amizades que me ajudaram a me manter são nesse mundo. Quero agradecer especialmente ao João, Higor, Augusto, Matheus, Victor, Daniel e a Sylvia, que, na medida do possível, se manteram próximos nos últimos anos e são muito importantes para mim. Sou feliz por ter conhecido cada um deles.

Por fim, eu gostaria de agradecer a Lorena, a pessoa que mais me conhece no mundo. Por mais de 6 anos de amor, amizade, carinho, cumplicidade, risadas e aventuras, eu serei eternamente grato.
