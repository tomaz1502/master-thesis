\documentclass[
	msc,
	english
%% Para dissertações de mestrado, OU
%	mscproposta, %% Para propostas de dissertação de mestrado, OU
%	phd, %% Para teses de doutorado, OU
%	phdproposta, %% Para propostas de tese de doutorado
%	portugues %% Para documentos em português, OU
%	english %% Para documentos em inglês
]{ppgccufmg}

%\usepackage[brazil]{babel} %% se o documento for em português, OU
\usepackage[english]{babel} %% se o documento for em inglês
%\usepackage[latin1]{inputenc}
\usepackage{natbib}
\usepackage{xcolor}
\usepackage{lipsum}
\usepackage[
	colorlinks=true,
	linkcolor=blue, %% Cor dos links do sumário
	citecolor=red, %% Cor dos links das citações      
	urlcolor=magenta, %% Cor das urls
]{hyperref}
\usepackage{minted}
\usepackage{algorithm}
\usepackage{algpseudocode}
\usepackage[toc]{glossaries}
\usepackage{glossaries-extra}
\usepackage{bm}

%% Exemplo de lista customizada ==================
%% Para criar uma lista customizada (como Lista de Algoritmos, Lista de Exemplos) que ficará juntamente com as Lista de Figuras e Lista de Tabelas, execute os 3 comandos abaixo substituindo "algoritmos" pelo tipo de lista que estará criando. Para adicionar a lista ao documento, deverá passar o seguinte parâmetro no comando \ppgccufmg:
%% \ppgccufmg{
%% 		...
%% 		listacustomizada={\listadealgoritmos}
%% }
% \newfloat[chapter]{algoritmo}{lol}{Algoritmo}
% \newcommand{\listaalgoritmosname}{Lista de Algoritmos} %% Título da lista
% \newlistof{listadealgoritmos}{lol}{\listaalgoritmosname} %% O primeiro parâmetro é o nome da lista, e este deverá ser passado no parâmetro listacustomizada={\nomedalista}
% \newlistentry{algoritmo}{lol}{0} %% Nome do ambiente de cada algoritmo, e.g., \begin{algoritmo} ... \end{algoritmo}

%% **** Caso não haja nenhuma lista adicional, os comandos acima podem ser apagados. ****
%% ===============================================

\newcommand{\yell}[1]{{\color{blue} [#1]}}
\newcommand{\tom}[1]{\yell{#1 --tom}}

\newglossarystyle{mystyle}{%
  \glossarystyle{long}%
  \renewenvironment{theglossary}%
     {\begin{longtable}{p{3cm}p{\glsdescwidth}}}%
     {\end{longtable}}%
}

\newglossary{symbols}{sym}{sbl}{List of Symbols}
\makeglossaries
\newglossaryentry{NegLit}
{
    name = {$\bm{\overline x}$},
    description={Negation of the literal $x$}
}
\newglossaryentry{IncVar}
{
  name = {$\bm{x \in C}$},
  description={Variable $x$ is present in the clause $C$}
}
\newglossaryentry{RemVar}
{
  name = {$\bm{C \setminus x}$},
  description={Clause $C$ without variable $x$}
}

\begin{document}
	\ppgccufmg{
		autor={Tomaz Gomes Mascarenhas}, %% Autor(a)
		titulopt={Demonstrando teoremas em Lean por meio da reconstrução de provas em SMT},
		tituloen={Proving Lean theorems via reconstructed SMT proofs}, %% Título em inglês
		cidade={Belo Horizonte},
		ano={2023},
		versaopt={Final},
		versaoen={Final}, %% Palavra que acompanhará 'Version' na folha de rosto em inglẽs
		orientador={Haniel Barbosa}, %% Para masculino
		% fichacatalografica={fichacatalografica.pdf},
		% folhadeaprovacao={folhadeaprovacao.pdf},
		resumo={resumo.tex}, %% Resumo em português
		palavraschave={Verifica\c{c}\~ao Formal, Lean, SMT},
		abstracten={abstract.tex}, %% Abstract em inglês
		keywords={Formal Verification, Lean, SMT}, %% Palavras-chave do abstract
		% dedicatoria={dedicatoria.tex}, %% Arquivo .tex contendo a dedicatória
		% agradecimentos={agradecimentos.tex},
		epigrafe={S\'o quem sonha acordado v\^e o sol nascer.},
		epigrafeautor={Unknown},
		% listadefiguras={sim}, %% Remova (ou comente) este parâmetro para remover a lista de figuras
		% listadetabelas={sim}, %% Remova (ou comente) este parâmetro para remover a lista de tabelas
		% listascustomizadas={\listadealgoritmos} %% Lista customizada (e.g., lista de algoritmos).
	}

    \printunsrtglossary[style=mystyle]

	\newpage

	\tableofcontents

	\chapter{Introduction}
	  \section{Context}
	    A mechanized proof is a proof, written in some language recognized by a computer, that had its validity checked by a trusted verifier. One of the main applications of these artifacts are formalizing mathematical theories. Indeed, for the former, there are well-known examples of successful formalizations. One of them is the mechanization of the proof of a theorem regarding Perfectoid Spaces\cite{scholze}, performed by the fields-medal mathematician Peter Scholze, together with the community of a system called Lean\cite{lean}. Scholze proved the theorem using pen and paper, but was unsure of the result due to its complexity. Once he translated the theorem and the proof to the language of Lean, the system could point out some mistakes he made, and, after fixing them, he could be sure of the correctness of the proof.

Another application of mechanized proofs is verifying the correctness of mission-critical software. Given a specification of the behavior of some program, the program is said to be correct if it respects the specification for any input it is given. There are a variety of techniques to obtain correctness evidence for a software. The most common one is the development of tests. Besides being easy to write an efficient set of tests, there are many types of bugs that can be discovered with its execution. In fact, this approach is enough for a large amount of problems that are solved by software engineering. However, tests can’t guarantee that a program doesn’t have flaws, since the number of valid inputs is almost always exceedingly large, or infinite. This kind of guarantee is extremely important for mission-critical software, that is, systems that have critical responsibilities, such as the control of airplanes or medical equipment. In this context, one promising alternative is to use a mechanized proof of the correctness of the software to guarantee its safety.

The process of generating mechanized proofs can be divided into
two categories: interactive and automatic.

Interactive theorem provers (ITPs) are mainly represented by proof assistants, in which, after defining
a theorem, the user attempts to manually write a proof for it,
relying on the tool to organize the set of hypothesis and
how the goal changed step-wisely through the proof, as well as to ensure the
correctness of each step according to a small, trusted kernel.
%
Each logic step must be explicitly stated by the user, which makes the tool
costly to be used.

Automatic theorem provers (ATPs), on the other hand,
only require the user to define a conjecture, proceeding automatically to
determine whether there exists a proof for it, or possibly providing a
counter-example if it can find one.
%
Although they are easier to use, ATPs require a large
codebase to implement all the algorithms necessary to execute the search for a proof,
making them more susceptible to errors and harder to be trusted, since the
larger is the codebase, the more complicated it is
to verify it, and, also, once it is verified, its development becomes freezed
(otherwise it would have to be verified again).

A common approach to address the trust issue for ATPs is to have them provide a
proof to support their results, so that it can be independently verified whether
it indeed proves the theorem in question.
%
Via these proofs the automatic proving performed by ATPs can be
leveraged by ITPs, since their requirement to accepting a proof, i.e.\ that each
step is correct according to its internal logic, can be applied to the ATP
proof.
%
By connecting these systems, it would be possible for the user to focus on more complex steps of the proof, such as defining an induction hypothesis, while delegating the burden of other long and straightforward steps to the ATP. Indeed, this
connection is so important that there are projects like Hammering Towards QED~\cite{hammering}
that outline all the efforts that were already made in order to integrate
interactive and automatic theorem provers. In this paper, the authors describe in detail each component that a system that creates a connection between ATPs and ITPs has to implement, as well as the main issues that they have to solve, based on existing programs that were successful in this task. Besides that, they show their potential through several large benchmarks. 


	  \section{Contributions}
	    We have built a system that takes proofs of the unsatisfiability of
SMT queries produced by the SMT solver cvc5~\cite{cvc5} and
reconstructs them as proofs in the ITP Lean 4~\cite{lean},
which has as its main goal the capacity of reconstructing the largest possible
range of proofs produced by cvc5.
%
We present in detail two strategies well-known in the literature for doing this translation.
%
We have effectively applied one of them, and we share our insights
regarding the reasons we believe it is more appropriate than the other
for our goals.

The main motivation of this project is that, despite the fact that Lean is
emerging as a promising programming language and ITP and being
widely used by mathematicians in large-scale
formalizations~\cite{scholze, mathlib}, there is currently no way to
interact with SMT solvers from it, even though these systems have been
central in previous developments of proof automation in ITPs, as we will show in Section~\ref{sec:related}. The contribution of the present work
is essential to develop this kind of project using Lean.

The cvc5 solver has a module for post-processing its proofs,
translating and printing them in a format recognized by Lean~\cite{Barbosa2022} (which is the reason why we chose it).
Each proof produced by cvc5 consists of a set of logical inferences starting from the
hypothesis until the goal is inferred. In order to reconstruct such proofs inside of
Lean, it is necessary to prove the correctness of these inference rules inside the
ITP.\
Our main contribution is to provide a set of tactics and theorems matching the set
of rules used by the ATP, enabling the verification of the translated proofs through Lean's kernel.
Furthermore, it is possible to leverage these proofs to discharge the
responsibility of proving theorems originally stated in Lean to cvc5, which is
the main use case for our module. There are other modules that need to be implemented
in order to fully automate this process (which we will describe in Section~\ref{sec:related}), that are out of the scope of the present project.
However, our project is being used as part of the joint project
LeanSMT\footnote{The code for the project can be found at \url{https://github.com/ufmg-smite/lean-smt}}, that aims to implement a tactic in Lean
that would perform this process of discharging proofs. Starting from a Lean goal, the tactic would translate
it to an SMT query, then invoke the solver to try to prove it and lift the proof produced
(in case it is found) to Lean's language, so that it can be used as a proof for the original
goal.

Another contribution of our work is the first checker for cvc5 proofs
(which are expressed in the theories we support) verified by an established ITP.\@
%
Other checkers exist (\cite{carcara} and~\cite{lfsc}, for instance), but one has to
trust their source code to verify proofs with them. While it is necessary
to trust Lean's kernel to verify proofs with our module, Lean is a tool
with thousands of users constantly writing proofs in it. Therefore,
if it had an inconsistency in its kernel, it is very likely that
an user would have already discovered it.
%
While this is an useful feature of our tool, it is important to highlight that
we did not design it as a proof checker and it lacks optimization for this purpose,
resulting in a suboptimal performance for this task.

% checking proofs is not its main use case and the tool was not optimized for this.

	  \section{Related Work}
	    \subsection{Hammering Towards QED}
           \label{sec:hammering}
			\label{sec:hammering}
As previously mentioned, Hammering Towards QED is a project
that aims to describe all the tools, which the paper calls ``hammers'',
that were created with the purpose of connecting automatic and interactive
theorem provers. Furthermore, this document also outlines
the main components that such tools usually have:

\begin{itemize}
  \item The premise selection module: it identifies
        a subset of the facts previously known by the
        ITP that are likely to be useful for the ATP to
        prove the given goals.\@
  \item The translation module: it uses the premisses selected and the original
        theorem from the ITP to formulate a query in the language of the ATP.\ This query
        must be equivalent to the original theorem, with all premisses selected
        added as hypothesis.
  \item The proof reconstruction module: it lifts the proof produced
        by the ATP into a proof that is accepted by the ITP.\@
\end{itemize}

In this dissertation we describe how we implemented a proof reconstruction module
for Lean users from cvc5 proofs.
The three main strategies used to reconstruct the proof produced
by the automatic system inside the interactive described in the paper are as follows:

The first one is known as the \textit{certified} approach~\cite{snipe}. In this case,
the hammer defines a datatype to represent
terms in the language of the ATP and a set of functions to manipulate
values of those datatypes, representing
the logical axioms that the solver uses to reason about those terms. Then, a lifting function is defined, that is,
a function that takes a value of this datatype and outputs an equivalent term in the native
language of the ITP.\ Finally, the correctness of each function is
verified with respect to the lifting function, in the sense that, if the input term
was lifted to a value that is true in the ITP's logic, then the output term will
also be true. The ATP's proof will be represented as a sequence of applications those
functions, and their correctness are proved \textit{a priori}. When checking
a specific proof, the only step that the hammer must perform is to compute the result
of the application of all the functions in the solver's output, and to check if
the final term matches with the expected one.

The second approach known as the \textit{certifying} approach~\cite{snipe}. It consists of matching
each axiom in the ATP's logic into tactics
defined in the ITP that operate directly over native terms of the system and parsing
the proof produced by the automatic solver into a sequence of applications
of those tactics, which are replayed inside the ITP.\
In this case, the proof steps are reconstructed and checked in the ITP on the
fly each time the hammer is invoked. Since there is no theorem stating the
correctness of them, it is possible that this process fails.
On the other hand,
this technique skips computations done over embedded terms, which have to be done by the
certified approach, having the potential to have a better
performance. It also diminishes the complexity of the tool, as there is no need to
define an intermediate representation inside the ITP.\

The third one is similar to the certifying approach, the only difference being that
the proof, after parsed into a sequence of applications of tactics, is stored in a separate
file instead of replayed by the ITP.\ After this, it's possible to run external tools
that perform postprocessing over this proof to simplify it~\cite{hammer_20, hammer_21}.
In some cases it's even possible to ignore a large
portion of the original proof. The proof is then replayed in the ITP, in the same way as the process
employed by the certifying approach. This technique can be inconvenient for very large proofs,
as it requires the script to be stored in the filesystem. However,
it has the advantage over the two previous methods of only requiring access to the ATP
on the first time that the proof is checked.

In this project we will be using the second approach. We give more details about this
decision in the later chapters. The next two sections describe examples of hammers.

	    \subsection{SMTCoq}
           \label{sec:smtcoq}
			SMTCoq~\cite{smtcoq} is a hammer for the Coq proof assistant~\cite{Bertot2004}.
It offers tactics to prove theorems in Coq via the external proof-producing SMT
solvers veriT~\cite{Bouton2009} and CVC4~\cite{Barrett2011}. The tool can support multiple proof formats
due to a preprocessor written in OCaml, that is able to turn them
into a unified certificate in the Coq language,
which will be used as input for the plugin. Our tool explores one specific
proof format of cvc5, therefore, for now, it only supports this solver and is
less flexible in this aspect than SMTCoq.

The core of the hammer directly follows the certified transformations approach
(although recently, it was extended with new features, implemented using
certifying transformations~\cite{snipe}).
%
It has a set of certified functions representing the transformations
that are sound in three of the main domains that are available in SMT:\
Linear Arithmetic, Uninterpreted Functions and Bitvectors. The first one is used to
represent formulas involving numeric variables, the second one is
used to represent congruence relatons over variables and uninterpreted functions and
the third one is used to simulate operations over numbers as they are represented
by computers, preserving the semantics of this representation.
%
Our tool currently supports Linear Arithmetic and Uninterpreted Functions.

The way that SMT solvers prove
that a proposition is true is by showing that the negation of the proposition is unsatisfiable,
that is, false for any assignment of its free variables. This kind of proof
is parsed in SMTCoq into a sequence of applications of certified functions, which have
to transform the term representing the negation of the original goal into a
term that will be lifted to the \textit{False} proposition in Coq. Once the
ITP verifies that the sequence of steps produced by the solver indeed
produces \textit{False}, the hammer apply its main theorem, which states that if
this process was successful, then the original goal is true.

SMTCoq also differs from our work as it implements a translation module, but
it does not implement a premise selection module.

		\subsection{Sledgehammer}
           \label{sec:sledgehammer}
			The ITP Isabelle/HOL~\cite{Nipkow2002} has a similar tool,
namely, Sledgehammer~\cite{sledgehammer}. This system achieves its goal by
invoking several SMT solvers in parallel to prove a given goal and collecting
their output to determine which lemmas must be applied in order to prove the theorem
inside Isabelle. In a way, this approach is very similar to the one we're using in this project, as the proof is produced on
the fly (known as the Certifying approach, which will also be described in Section~\ref{sec:certifiedVsCertifying}) as opposed
to having a single theorem that establishes once and for all
that, if all steps performed by the solver were successful,
then the original goal is valid, as is done by SMTCoq.

	  \section{Organization of this document}
	    In Chapter~\ref{chap:formalPrelim} we present the concepts involved in our work.
We describe the SMT problem in detail, as well as the main techniques used to solve
it and the proof certificates produced by cvc5. We also present Lean and give
an overview of the features of the language we have employed throughout the project.
%
Then, in Chapter~\ref{chap:certified}, we describe the steps necessary to implement
the reconstruction of proofs from cvc5 in Lean using the approach of certified
transformations.
%
Next, in Chapter~\ref{chap:rcons}, we present in detail our implementation of the
reconstruction using the certifying transformations approach.
%
This implementation is evaluated in Chapter~\ref{chap:eval}.
%
Finally, we present some possible directions for future work in
Chapter~\ref{chap:future}.

	\chapter{Formal Preliminaries}
	  \section{Satisfiability Modulo Theories}
	    
\subsection{Description of the Problem}

Satisfiability Modulo Theories (SMT)~\cite{smt} is a generalization of the Boolean
Satisfiability Problem (SAT). In this version, the underlying logic is First Order Logic
instead of Propositional Logic, that is, the input formula for some instance of the
problem can contain quantifiers binding variables which will affect the satisfiability of
that instance. Another addition is the inclusion of a set of theories that allows the
problem to refer to
variables of different domains. More precisely, a theory consists of a sort (for instance, integers) over which a subset of the variables of the problem can range over and a set of operations (for instance, addition and comparison operations) can be applied to those variables.

TODO:
\begin{itemize}
  \item give examples to make it more clear
  \item motivate
  \item define MSFOL like bohme?
\end{itemize}


\subsection{SMT-LIB ?}

\subsection{SMT Solvers}

	  \section{Lean}
	    take a look at chapter 2 of smtcoq
	    \begin{itemize}
	      \item Falar sobre porque eh facil confiar no proof assistant
		  \item Explicar que taticas extendem a linguagem mas nao aumentam o trusted core
	    \end{itemize}
	  \section{Lean's Framework for Metaprogramming}
	\chapter{Certifying Reconstruction of SMT Proofs in Lean}
	  \section{Certified vs Certifying}
      \label{sec:certifiedVsCertifying}
	  \section{Classical vs Intuitionist (?)}
	  \section{Tactics}
	  \section{The Complete Architecture}
	  \section{Skipping the Parser}
	  % maybe think of other optimizations and make a section about this
	\chapter{Evaluation}
	\chapter{Future Work}

	% \chapter{Introdução}
	% 	a introducao vem aqui
	% \chapter{Desenvolvimento}
	% 	\lipsum[1-4]

	% 	\begin{algoritmo}
	% 		\centering
	% 		\framebox[.5\textwidth]{\texttt{Código, Código, Código}}
	% 		\caption{Este é o meu Algoritmo 1.}
	% 		\label{alg:algoritmo1}
	% 	\end{algoritmo}

	% 	\begin{algoritmo}
	% 		\centering
	% 		\framebox[.7\textwidth]{\texttt{Mais Código, Mais Código, Mais Código}}
	% 		\caption{Este é o meu Algoritmo 2.}
	% 		\label{alg:algoritmo2}
	% 	\end{algoritmo}

	% 	\begin{figure}[h]
	% 		\centering
	% 		\includegraphics[width=\textwidth]{img/dcc.jpg}
	% 		\caption{Prédio do DCC em 2016.}
	% 		\label{fig:exemplo}
	% 	\end{figure}

	% 	\lipsum[5]

	% 	\begin{table}[h]
	% 		\centering
	% 		\begin{tabular}{c|ccccccccl}
	% 			Natural & \multicolumn{9}{c}{Real}   \\ \hline
	% 			1 & 0.  & {\color{red} 2}  & 3   & 6   & 4   & 3   & 6   & 7   & $\ldots$ \\
	% 			2  & 0.  & 0   & {\color{red} 9}  & 8   & 4   & 7   & 3   & 2   & $\ldots$ \\
	% 			3  & 0.  & 1   & 9   & {\color{red} 3}  & 2   & 1   & 4   & 0   & $\ldots$ \\
	% 			4  & 0.  & 8   & 4   & 3   & {\color{red} 2}  & 7   & 9   & 2   & $\ldots$ \\
	% 			5  & 0.  & 0   & 1   & 2   & 9   & {\color{red} 3}  & 4   & 8   & $\ldots$ \\
	% 			6  & 0.  & 2   & 8   & 2   & 6   & 5   & {\color{red} 8}  & 3   & $\ldots$ \\
	% 			7  & 0.  & 0   & 2   & 1   & 5   & 3   & 7   & {\color{red} 4}  & $\ldots$ \\
	% 			$\vdots$ & $\vdots$  & $\vdots$  & $\vdots$  & $\vdots$  & $\vdots$  & $\vdots$  & $\vdots$  & $\vdots$  & $\ddots$ \\ \hline
	% 			\multicolumn{1}{l|}{} & \multicolumn{1}{l}{0.} & \multicolumn{1}{l}{{\color{red} 2}} & \multicolumn{1}{l}{{\color{red} 9}} & \multicolumn{1}{l}{{\color{red} 3}} & \multicolumn{1}{l}{{\color{red} 2}} & \multicolumn{1}{l}{{\color{red} 3}} & \multicolumn{1}{l}{{\color{red} 8}} & \multicolumn{1}{l}{{\color{red} 4}} & $\ldots$
	% 		\end{tabular}
	% 		\caption{Cantor: Existem infinitos diferentes!}
	% 		\label{tab:exemplo}
	% 	\end{table}

	% 	\section{Usando referências}
	% 		Segundo \cite{horn86robot}, todo triângulo equilátero tem os lados iguais. Já segundo \cite{shashua97photometric}, todo quadrado também tem.

	% 		Veja que o pacote \verb|natbib| permite uma série de formas diferentes para fazer referências bibliográficas. O comando padrão, \verb|\cite|, realiza a citação comum vista no parágrafo anterior. Outros comandos permitem, por exemplo, colocar automaticamente a citação entre	parênteses \citep{hougen93estimation, sato99illumination2, sato99illumination1, sato01stability}.

	% 		O comando usado foi \verb|\citep|. Veja a documentação do \verb|natbib| na Internet para conhecer	outros comandos e exemplos de uso.

	% 		Citações aleatórias para fazer com que as referências bibliográficas ocupem	mais de uma página: \cite{bichsel92simple, dror01statistics, guisser92new, dwork2006calibrating, sweeney2002k}.

		%% Referências
		\bibliographystyle{plain}
		\bibliography{referencias}

		% \begin{apendices}
		% 	\chapter{Um apêndice}
		% 		\lipsum[1-3]

		% 	\chapter{Outro Apêndice}
		% 		\lipsum[4-6]

		% \end{apendices}
		
\end{document}
