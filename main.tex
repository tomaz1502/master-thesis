\documentclass[
	msc,
	english
%% Para dissertações de mestrado, OU
%	mscproposta, %% Para propostas de dissertação de mestrado, OU
%	phd, %% Para teses de doutorado, OU
%	phdproposta, %% Para propostas de tese de doutorado
%	portugues %% Para documentos em português, OU
%	english %% Para documentos em inglês
]{ppgccufmg}

%\usepackage[brazil]{babel} %% se o documento for em português, OU
\usepackage[english]{babel} %% se o documento for em inglês
%\usepackage[latin1]{inputenc}
\usepackage{natbib}
\usepackage{xcolor}
\usepackage{lipsum}
\usepackage[
	colorlinks=true,
	linkcolor=blue, %% Cor dos links do sumário
	citecolor=red, %% Cor dos links das citações      
	urlcolor=magenta, %% Cor das urls
]{hyperref}
\usepackage{amsmath}
\usepackage{amssymb}
\usepackage{minted}
\usepackage{algorithm}
\usepackage{algpseudocode}
\usepackage[toc]{glossaries}
\usepackage{glossaries-extra}
\usepackage{bm}
\usepackage{tikz}
\usepackage{relsize}
\usepackage[normalem]{ulem}
\usepackage{stmaryrd}
\usepackage{proof}
\usepackage{mathtools}
\usepackage{amsfonts}
\usepackage{amsthm}
% \usepackage[proofrulebaseline=2ex]{prooftrees}


%% Exemplo de lista customizada ==================
%% Para criar uma lista customizada (como Lista de Algoritmos, Lista de Exemplos) que ficará juntamente com as Lista de Figuras e Lista de Tabelas, execute os 3 comandos abaixo substituindo "algoritmos" pelo tipo de lista que estará criando. Para adicionar a lista ao documento, deverá passar o seguinte parâmetro no comando \ppgccufmg:
%% \ppgccufmg{
%% 		...
%% 		listacustomizada={\listadealgoritmos}
%% }
% \newfloat[chapter]{algoritmo}{lol}{Algoritmo}
% \newcommand{\listaalgoritmosname}{Lista de Algoritmos} %% Título da lista
% \newlistof{listadealgoritmos}{lol}{\listaalgoritmosname} %% O primeiro parâmetro é o nome da lista, e este deverá ser passado no parâmetro listacustomizada={\nomedalista}
% \newlistentry{algoritmo}{lol}{0} %% Nome do ambiente de cada algoritmo, e.g., \begin{algoritmo} ... \end{algoritmo}

%% **** Caso não haja nenhuma lista adicional, os comandos acima podem ser apagados. ****
%% ===============================================

\DeclareUnicodeCharacter{2200}{$\mathbb{\forall}$}

\newtheorem{theorem}{Theorem}
\newtheorem{definition}{Definition}
\newtheorem{example}{Example}

\newcommand{\yell}[1]{{\color{blue} [#1]}}
\newcommand{\tom}[1]{\yell{#1 --tom}}

\newglossarystyle{mystyle}{%
  \glossarystyle{long}%
  \renewenvironment{theglossary}%
     {\begin{longtable}{p{3cm}p{\glsdescwidth}}}%
     {\end{longtable}}%
}

\newglossary{symbols}{sym}{sbl}{List of Symbols}
\makeglossaries
% \newglossaryentry{NegLit}
% {
%     name = {$\bm{\overline x}$ },
%     description={Negation of the literal $x$}
% }
\newglossaryentry{IncVar}
{
  name = {$\bm{x \in C}$ },
  description={Variable $x$ is present in the clause $C$}
}
\newglossaryentry{RemVar}
{
  name = {$\bm{C \setminus x}$ },
  description={Clause $C$ without variable $x$}
}
\newglossaryentry{FormulaVars}
{
  name = {$\bm{\textit{Vars}(\psi)}$ },
  description={Set of all variables in the formula $\psi$}
}
\newglossaryentry{Resolution}
{
  name = {\bm{$C_{1} \diamond_{x} C_{2}$} },
  description={Clause resulting from applying resolution with $C_{1}$ and $C_{2}$ using $x$ as a pivot}
}
\newglossaryentry{VarAssign}
{
  name = {$\bm{\psi_{\{x \gets val\}}$} },
  description = {Formula resulting of assigning the variable $x$ to the boolean value $val$ in $\psi$}
}

\begin{document}
	\ppgccufmg{
		autor={Tomaz Mascarenhas}, %% Autor(a)
		titulopt={Demonstrando teoremas em Lean por meio da reconstrução de demonstra\c{c}\~oes em SMT},
		tituloen={Proving Lean theorems via reconstructed SMT proofs}, %% Título em inglês
		cidade={Belo Horizonte},
		ano={2023},
		versaopt={Final},
		versaoen={Final}, %% Palavra que acompanhará 'Version' na folha de rosto em inglẽs
		orientador={Haniel Barbosa}, %% Para masculino
		% fichacatalografica={fichacatalografica.pdf},
		% folhadeaprovacao={folhadeaprovacao.pdf},
		resumo={resumo.tex}, %% Resumo em português
		palavraschave={Verifica\c{c}\~ao Formal, Lean, SMT},
		abstracten={abstract.tex}, %% Abstract em inglês
		keywords={Formal Verification, Lean, SMT}, %% Palavras-chave do abstract
		% dedicatoria={dedicatoria.tex}, %% Arquivo .tex contendo a dedicatória
		% agradecimentos={agradecimentos.tex},
		epigrafe={S\'o quem sonha acordado v\^e o sol nascer.},
		epigrafeautor={Unknown},
		% listadefiguras={sim}, %% Remova (ou comente) este parâmetro para remover a lista de figuras
		% listadetabelas={sim}, %% Remova (ou comente) este parâmetro para remover a lista de tabelas
		% listascustomizadas={\listadealgoritmos} %% Lista customizada (e.g., lista de algoritmos).
	}

    \printunsrtglossary %[style=mystyle]

	\newpage

	\tableofcontents

	\chapter{Introduction}
	  \section{Context}
	    A mechanized proof is a proof, written in some language recognized by a computer, that had its validity checked by a trusted verifier. One of the main applications of these artifacts are formalizing mathematical theories. Indeed, for the former, there are well-known examples of successful formalizations. One of them is the mechanization of the proof of a theorem regarding Perfectoid Spaces\cite{scholze}, performed by the fields-medal mathematician Peter Scholze, together with the community of a system called Lean\cite{lean}. Scholze proved the theorem using pen and paper, but was unsure of the result due to its complexity. Once he translated the theorem and the proof to the language of Lean, the system could point out some mistakes he made, and, after fixing them, he could be sure of the correctness of the proof.

Another application of mechanized proofs is verifying the correctness of mission-critical software. Given a specification of the behavior of some program, the program is said to be correct if it respects the specification for any input it is given. There are a variety of techniques to obtain correctness evidence for a software. The most common one is the development of tests. Besides being easy to write an efficient set of tests, there are many types of bugs that can be discovered with its execution. In fact, this approach is enough for a large amount of problems that are solved by software engineering. However, tests can’t guarantee that a program doesn’t have flaws, since the number of valid inputs is almost always exceedingly large, or infinite. This kind of guarantee is extremely important for mission-critical software, that is, systems that have critical responsibilities, such as the control of airplanes or medical equipment. In this context, one promising alternative is to use a mechanized proof of the correctness of the software to guarantee its safety.

The process of generating mechanized proofs can be divided into
two categories: interactive and automatic.

Interactive theorem provers (ITPs) are mainly represented by proof assistants, in which, after defining
a theorem, the user attempts to manually write a proof for it,
relying on the tool to organize the set of hypothesis and
how the goal changed step-wisely through the proof, as well as to ensure the
correctness of each step according to a small, trusted kernel.
%
Each logic step must be explicitly stated by the user, which makes the tool
costly to be used.

Automatic theorem provers (ATPs), on the other hand,
only require the user to define a conjecture, proceeding automatically to
determine whether there exists a proof for it, or possibly providing a
counter-example if it can find one.
%
Although they are easier to use, ATPs require a large
codebase to implement all the algorithms necessary to execute the search for a proof,
making them more susceptible to errors and harder to be trusted, since the
larger is the codebase, the more complicated it is
to verify it, and, also, once it is verified, its development becomes freezed
(otherwise it would have to be verified again).

A common approach to address the trust issue for ATPs is to have them provide a
proof to support their results, so that it can be independently verified whether
it indeed proves the theorem in question.
%
Via these proofs the automatic proving performed by ATPs can be
leveraged by ITPs, since their requirement to accepting a proof, i.e.\ that each
step is correct according to its internal logic, can be applied to the ATP
proof.
%
By connecting these systems, it would be possible for the user to focus on more complex steps of the proof, such as defining an induction hypothesis, while delegating the burden of other long and straightforward steps to the ATP. Indeed, this
connection is so important that there are projects like Hammering Towards QED~\cite{hammering}
that outline all the efforts that were already made in order to integrate
interactive and automatic theorem provers. In this paper, the authors describe in detail each component that a system that creates a connection between ATPs and ITPs has to implement, as well as the main issues that they have to solve, based on existing programs that were successful in this task. Besides that, they show their potential through several large benchmarks. 


	  \section{Contributions}
	    We have built a system that takes proofs of the unsatisfiability of
SMT queries produced by the SMT solver cvc5~\cite{cvc5} and
reconstructs them as proofs in the ITP Lean 4~\cite{lean},
which has as its main goal the capacity of reconstructing the largest possible
range of proofs produced by cvc5.
%
We present in detail two strategies well-known in the literature for doing this translation.
%
We have effectively applied one of them, and we share our insights
regarding the reasons we believe it is more appropriate than the other
for our goals.

The main motivation of this project is that, despite the fact that Lean is
emerging as a promising programming language and ITP and being
widely used by mathematicians in large-scale
formalizations~\cite{scholze, mathlib}, there is currently no way to
interact with SMT solvers from it, even though these systems have been
central in previous developments of proof automation in ITPs, as we will show in Section~\ref{sec:related}. The contribution of the present work
is essential to develop this kind of project using Lean.

The cvc5 solver has a module for post-processing its proofs,
translating and printing them in a format recognized by Lean~\cite{Barbosa2022} (which is the reason why we chose it).
Each proof produced by cvc5 consists of a set of logical inferences starting from the
hypothesis until the goal is inferred. In order to reconstruct such proofs inside of
Lean, it is necessary to prove the correctness of these inference rules inside the
ITP.\
Our main contribution is to provide a set of tactics and theorems matching the set
of rules used by the ATP, enabling the verification of the translated proofs through Lean's kernel.
Furthermore, it is possible to leverage these proofs to discharge the
responsibility of proving theorems originally stated in Lean to cvc5, which is
the main use case for our module. There are other modules that need to be implemented
in order to fully automate this process (which we will describe in Section~\ref{sec:related}), that are out of the scope of the present project.
However, our project is being used as part of the joint project
LeanSMT\footnote{The code for the project can be found at \url{https://github.com/ufmg-smite/lean-smt}}, that aims to implement a tactic in Lean
that would perform this process of discharging proofs. Starting from a Lean goal, the tactic would translate
it to an SMT query, then invoke the solver to try to prove it and lift the proof produced
(in case it is found) to Lean's language, so that it can be used as a proof for the original
goal.

Another contribution of our work is the first checker for cvc5 proofs
(which are expressed in the theories we support) verified by an established ITP.\@
%
Other checkers exist (\cite{carcara} and~\cite{lfsc}, for instance), but one has to
trust their source code to verify proofs with them. While it is necessary
to trust Lean's kernel to verify proofs with our module, Lean is a tool
with thousands of users constantly writing proofs in it. Therefore,
if it had an inconsistency in its kernel, it is very likely that
an user would have already discovered it.
%
While this is an useful feature of our tool, it is important to highlight that
we did not design it as a proof checker and it lacks optimization for this purpose,
resulting in a suboptimal performance for this task.

% checking proofs is not its main use case and the tool was not optimized for this.

	  \section{Related Work}
	    \subsection{Hammering Towards QED}
           \label{sec:hammering}
			\label{sec:hammering}
As previously mentioned, Hammering Towards QED is a project
that aims to describe all the tools, which the paper calls ``hammers'',
that were created with the purpose of connecting automatic and interactive
theorem provers. Furthermore, this document also outlines
the main components that such tools usually have:

\begin{itemize}
  \item The premise selection module: it identifies
        a subset of the facts previously known by the
        ITP that are likely to be useful for the ATP to
        prove the given goals.\@
  \item The translation module: it uses the premisses selected and the original
        theorem from the ITP to formulate a query in the language of the ATP.\ This query
        must be equivalent to the original theorem, with all premisses selected
        added as hypothesis.
  \item The proof reconstruction module: it lifts the proof produced
        by the ATP into a proof that is accepted by the ITP.\@
\end{itemize}

In this dissertation we describe how we implemented a proof reconstruction module
for Lean users from cvc5 proofs.
The three main strategies used to reconstruct the proof produced
by the automatic system inside the interactive described in the paper are as follows:

The first one is known as the \textit{certified} approach~\cite{snipe}. In this case,
the hammer defines a datatype to represent
terms in the language of the ATP and a set of functions to manipulate
values of those datatypes, representing
the logical axioms that the solver uses to reason about those terms. Then, a lifting function is defined, that is,
a function that takes a value of this datatype and outputs an equivalent term in the native
language of the ITP.\ Finally, the correctness of each function is
verified with respect to the lifting function, in the sense that, if the input term
was lifted to a value that is true in the ITP's logic, then the output term will
also be true. The ATP's proof will be represented as a sequence of applications those
functions, and their correctness are proved \textit{a priori}. When checking
a specific proof, the only step that the hammer must perform is to compute the result
of the application of all the functions in the solver's output, and to check if
the final term matches with the expected one.

The second approach known as the \textit{certifying} approach~\cite{snipe}. It consists of matching
each axiom in the ATP's logic into tactics
defined in the ITP that operate directly over native terms of the system and parsing
the proof produced by the automatic solver into a sequence of applications
of those tactics, which are replayed inside the ITP.\
In this case, the proof steps are reconstructed and checked in the ITP on the
fly each time the hammer is invoked. Since there is no theorem stating the
correctness of them, it is possible that this process fails.
On the other hand,
this technique skips computations done over embedded terms, which have to be done by the
certified approach, having the potential to have a better
performance. It also diminishes the complexity of the tool, as there is no need to
define an intermediate representation inside the ITP.\

The third one is similar to the certifying approach, the only difference being that
the proof, after parsed into a sequence of applications of tactics, is stored in a separate
file instead of replayed by the ITP.\ After this, it's possible to run external tools
that perform postprocessing over this proof to simplify it~\cite{hammer_20, hammer_21}.
In some cases it's even possible to ignore a large
portion of the original proof. The proof is then replayed in the ITP, in the same way as the process
employed by the certifying approach. This technique can be inconvenient for very large proofs,
as it requires the script to be stored in the filesystem. However,
it has the advantage over the two previous methods of only requiring access to the ATP
on the first time that the proof is checked.

In this project we will be using the second approach. We give more details about this
decision in the later chapters. The next two sections describe examples of hammers.

	    \subsection{SMTCoq}
           \label{sec:smtcoq}
			SMTCoq~\cite{smtcoq} is a hammer for the Coq proof assistant~\cite{Bertot2004}.
It offers tactics to prove theorems in Coq via the external proof-producing SMT
solvers veriT~\cite{Bouton2009} and CVC4~\cite{Barrett2011}. The tool can support multiple proof formats
due to a preprocessor written in OCaml, that is able to turn them
into a unified certificate in the Coq language,
which will be used as input for the plugin. Our tool explores one specific
proof format of cvc5, therefore, for now, it only supports this solver and is
less flexible in this aspect than SMTCoq.

The core of the hammer directly follows the certified transformations approach
(although recently, it was extended with new features, implemented using
certifying transformations~\cite{snipe}).
%
It has a set of certified functions representing the transformations
that are sound in three of the main domains that are available in SMT:\
Linear Arithmetic, Uninterpreted Functions and Bitvectors. The first one is used to
represent formulas involving numeric variables, the second one is
used to represent congruence relatons over variables and uninterpreted functions and
the third one is used to simulate operations over numbers as they are represented
by computers, preserving the semantics of this representation.
%
Our tool currently supports Linear Arithmetic and Uninterpreted Functions.

The way that SMT solvers prove
that a proposition is true is by showing that the negation of the proposition is unsatisfiable,
that is, false for any assignment of its free variables. This kind of proof
is parsed in SMTCoq into a sequence of applications of certified functions, which have
to transform the term representing the negation of the original goal into a
term that will be lifted to the \textit{False} proposition in Coq. Once the
ITP verifies that the sequence of steps produced by the solver indeed
produces \textit{False}, the hammer apply its main theorem, which states that if
this process was successful, then the original goal is true.

SMTCoq also differs from our work as it implements a translation module, but
it does not implement a premise selection module.

		\subsection{Sledgehammer}
           \label{sec:sledgehammer}
			The ITP Isabelle/HOL~\cite{Nipkow2002} has a similar tool,
namely, Sledgehammer~\cite{sledgehammer}. This system achieves its goal by
invoking several SMT solvers in parallel to prove a given goal and collecting
their output to determine which lemmas must be applied in order to prove the theorem
inside Isabelle. In a way, this approach is very similar to the one we're using in this project, as the proof is produced on
the fly (known as the Certifying approach, which will also be described in Section~\ref{sec:certifiedVsCertifying}) as opposed
to having a single theorem that establishes once and for all
that, if all steps performed by the solver were successful,
then the original goal is valid, as is done by SMTCoq.

	  \section{Organization of this document}
	    In Chapter~\ref{chap:formalPrelim} we present the concepts involved in our work.
We describe the SMT problem in detail, as well as the main techniques used to solve
it and the proof certificates produced by cvc5. We also present Lean and give
an overview of the features of the language we have employed throughout the project.
%
Then, in Chapter~\ref{chap:certified}, we describe the steps necessary to implement
the reconstruction of proofs from cvc5 in Lean using the approach of certified
transformations.
%
Next, in Chapter~\ref{chap:rcons}, we present in detail our implementation of the
reconstruction using the certifying transformations approach.
%
This implementation is evaluated in Chapter~\ref{chap:eval}.
%
Finally, we present some possible directions for future work in
Chapter~\ref{chap:future}.

	\chapter{Formal Preliminaries}
	  \section{Satisfiability Modulo Theories (SMT)}
	    
\subsection{Description of the Problem}

Satisfiability Modulo Theories (SMT)~\cite{smt} is a generalization of the Boolean
Satisfiability Problem (SAT). In this version, the underlying logic is First Order Logic
instead of Propositional Logic, that is, the input formula for some instance of the
problem can contain quantifiers binding variables which will affect the satisfiability of
that instance. Another addition is the inclusion of a set of theories that allows the
problem to refer to
variables of different domains. More precisely, a theory consists of a sort (for instance, integers) over which a subset of the variables of the problem can range over and a set of operations (for instance, addition and comparison operations) can be applied to those variables.

TODO:
\begin{itemize}
  \item give examples to make it more clear
  \item motivate
  \item define MSFOL like bohme?
\end{itemize}


\subsection{SMT-LIB ?}

\subsection{SMT Solvers}

	  \section{Lean}
      \tom{disclaimer: for now I will just give a minimal definition on Lean's features. Later, if I discover that I need something else in the next chapter I add it here}

Lean is both an Interactive Theorem Prover and a programming
language. It is based on the Calculus of Inductive Constructions (CIC)~\cite{cic_ref} and explores the well-known correspondence between types and propositions~\cite{ch_correspondence} to implement a system that is both a proof checker and a type checker. This way, the ITP has a kernel with less than 7500 lines\footnote{Information obtained in 25/07/2023} that is powerful enough to recognize a language capable of expressing the theory of dependent types~\cite{dep_type_theory}.

\subsection{Lean as a Programming Language}

Lean has all the main features one can expect from a functional programming language. Its features include algebraic datatypes, pattern matching, polymorphism, typeclasses, IO support using monads and a robust macro system. The following script is a valid Lean program, that defines a new type representing natural numbers, together with a function for adding them:

\begin{minted}{lean}
inductive Natural where
  | zero : Natural
  | succ : Natural -> Natural

open Natural

def add (n m : Natural) : Natural :=
  match n with
  | zero    => m
  | succ n' => succ (add n' m)

notation x " + " y => add x y
\end{minted}

The keyword \textit{inductive} is used to introduce a new type, which in this case will be named \textit{Natural}. After the name, the user must use the keyword \textit{where}, followed by its constructors and their types. The constructors for the type Natural are \textit{zero} and \textit{succ}. This declaration introduces the constructors in the context with the names \textit{Natural.zero} and \textit{Natural.succ}. In order to be able to write just ``zero'' and ``succ'' we use the command \textit{open Natural}.

The next three lines define the sum function. We define new functions using the keyword \textit{def} followed by its name, the list of parameters and their types, the return type and the body of the function after the symbol ``:=''. In this case, we define the function by pattern matching on the first parameter. If it is \textit{zero}, we just return the second parameter. If it is \textit{succ} of some other value \textit{n'}, we return the \textit{succ} constructor applied to the result of a recursive call of \textit{add}, using \textit{n'} and \textit{m}.

Lastly, we use the command \textit{notation} to define a macro for using the add function with the usual infix ``+'' operator.

% Another useful feature of Lean is the possibility of asking for a type of a given expression. We do this with the command \textit{\#check}, followed by the expression we want to consult the type:

% \begin{minted}{lean}
%   #check add -- Natural -> Natural -> Natural
%   #check Natural -- Type
% \end{minted}

\subsection{Lean as a Theorem Prover}

In Lean, propositions are represented by types, which are inhabited by terms that represent proofs of those propositions. For instance, the following Lean expression represents a proposition (which is also a type) stating that, according to our previous definition of natural numbers, the addition of any natural \textit{n} and \textit{zero} results in \textit{n}:

\begin{minted}{lean}
∀ (n : Natural), (n + zero) = n
\end{minted}

Therefore, proving that this statement holds amounts to finding a term with this type. The following snippet shows the construction of such term:

\begin{minted}{lean}
theorem add_zero : ∀ (n : Natural), (n + zero) = n :=
  fun n =>
    match n with
    | zero    => rfl
    | succ n' => congrArg succ (add_zero n')
\end{minted}

This structure follows the same pattern as the one for defining functions, the only difference is the change of the keyword \textit{def} to \textit{theorem}. After the symbol ``:='' we have essentially a constructive proof of the statement. First, it introduces the variable binded by the $\forall$ symbol in the context, using \textit{fun n}. Then, it uses pattern match on \textit{n}. If \textit{n} is \textit{zero}, the type that is required is reduced to \textit{zero + zero = zero}, which follows directly from the definition of \textit{add}. The term \textit{rfl} is a proof that any term is equal to itself. In this case, Lean can match its type with the required one. If \textit{n} follows the pattern \textit{succ n'}, then the required type is \textit{succ n' + zero = succ n'}. By the definition of \textit{add}, the left-hand side evaluates to \textit{succ (n' + zero)}.
Note that the term \textit{add\_zero n'} has type \textit{n' + zero = n'}, which is almost the required one, missing only the \textit{succ} on both sides. This is solved by applying \textit{congrArg succ} on \textit{add\_zero n'} (\textit{congrArg} is a version of Theorem~\ref{cong_theorem} in Lean's library), which produces a term with the correct type.


We also present a proof that our addition function is commutative, together with another necessary lemma:

\begin{minted}{lean}
  theorem add_succ : ∀ (n m : Natural),
    (n + succ m) = succ (n + m) := fun n m =>
  match n with
  | zero    => rfl
  | succ n' => congrArg succ (add_succ n' m)

theorem add_comm : ∀ (n m : Natural), (n + m) = (m + n) := fun n m =>
  match n with
  | zero    => Eq.symm (add_zero m)
  | succ n' =>
    Eq.trans (congrArg succ (add_comm n' m)) (Eq.symm (add_succ m n'))
\end{minted}

Note that, since we have to make explicit every single logic step, even simple proofs are not easy to write and read. Lean (and most ITPs) provide an alternative for this kind of proof: the usage of tactics. As previously explained, tactics are routines that simulate common proof techniques. While we are building a proof term, Lean's kernel always keep track of a context, containing all declarations in scope and what is the currently expected type for the term we are building (also known as the goal). Tactics operate by manipulating these two structures, without compromising the trusted kernel. In other words, any modification that is made by a tactic must be properly justified and will be checked by Lean's kernel, in the same way it checked that our proof was correct.

We present a new version of \textit{add\_comm} with this new approach:

\begin{minted}{lean}
theorem add_comm' : ∀ (n m : Natural), (n + m) = (m + n) := by
  intros n m
  induction n with
  | zero       => rw [add, add_zero]
  | succ n' IH => rw [add, add_succ, IH]
\end{minted}

The keyword \textit{by} is used to communicate that the term will not be build explicitly, but instead computed by a tactic. Given a goal of the form $\forall (x : t) . P(x)$, the tactic \textit{intros x} will change the goal to $P(x)$ and introduce in the context a variable with name $x$ with type $t$. It can also be used to introduce multiple variables at one. We use it to introduce the two naturals, \textit{n} and \textit{m}. Next, we use the \textit{induction} tactic. This is a very general tactic that can apply the induction principle on any inductive type. Since our goal is of the form $P(n, m)$, where $n$ is a natural number, it will produce two new goals: $P(zero, m)$ and $P(n', m) \rightarrow P(succ(n'), m)$. Each one of this goals is then completed by the \textit{rw} tactic. Given a term that represents a proof of an equality $e_{1} = e_{2}$, the tactic \textit{rw} rewrites all occurrences of $e_{1}$ by $e_{2}$ in the goal. With this, we can avoid the usage of transitivity, symmetry and congruence lemmas that were needed on the other version of this proof. Note that this tactic also accepts function names such as \textit{add} as a parameter. In this case, it rewrites the definition of the function.

\subsection{Metaprogramming in Lean}

	\chapter{Proof Reconstruction}
    In this chapter we show how to lift proofs produced by cvc5 into
proofs of Lean statements using the certified transformations approach.
First, in Section~\ref{sec:gen-scripts}, we present the format used by
cvc5 to print its proofs as Lean scripts.
Then, in Section~\ref{sec:certified_rcons}, we outline the necessary
steps to lift these scripts into a sequence of certified transformations,
which will have its validity checked by Lean's kernel\footnote{Our implementation of this lifting can be found at \url{https://github.com/tomaz1502/lean-smt/tree/main/Smt/Reconstruction/Certified}}.
Lastly, in Section~\ref{sec:downsides}, we present some downsides of this approach.

    \section{Generating Scripts}
    As previously explained, cvc5 has a module for exporting its proofs as Lean scripts.
We will illustrate the format used by those scripts generated with an instance of the SMT problem corresponding
to the negation of modus ponens, i.e. $\neg (p \rightarrow ((p \rightarrow q) \rightarrow q))$.
We present this instance using SMT-Lib~\cite{smtlib}, a standardized syntax for
representing SMT problems recognized by most SMT solvers:

\begin{minted}{smtlib2.py -x}
(set-logic QF_UF)
(declare-const p Bool)
(declare-const q Bool)
(assert (not (=> p (=> (=> p q) q))))
\end{minted}

Feeding cvc5 with this script yields the result ``unsat'', as expected. The proof produced by the solver is shown in Figure~\ref{fig:cvc5-proof} and is an instance of the resolution tree introduced in Section~\ref{sec:proof_cert}. In addition to the resolution rule (in purple and green), there are also rules for conjunctive normal form transformations (yellow)\footnote{The rules in the figure are in the internal calculus of cvc5, which is documented in \url{https://cvc5.github.io/docs/latest/proofs/proof_rules.html}}. Note that all leaves (in blue) correspond to the input formula, \texttt{(not (=> p (=> (=> p q) q)))}.

\makeatletter
\setlength{\@fptop}{0pt}
\makeatother

\begin{figure}[t!]
  \centering
  \scalebox{0.45}{%
    \includegraphics[scale=0.9]{img/mp_cvc5_proof.png}}
  \caption{A cvc5 proof for the validity of Modus Ponens.}
  \label{fig:cvc5-proof}
\end{figure}

The encoding of this proof into Lean requires representing the terms that appear
in it, i.e.the formulas built with $p$ and $q$, as well as the rules.
%
This encoding uses a \emph{deep embedding} of the proof calculus of cvc5 into
Lean\footnote{The code is available at \url{https://github.com/tomaz1502/signatures/blob/smallCheckers/Cdclt/Lift/Other/BoolsExample.lean}}.
%
It provides a \texttt{term} type that models terms and formulas from MSFOL. This type has
a \texttt{const} constructor for defining symbols, parameterized with an
identifier (a natural number) and a \texttt{sort} (another type for encoding MSFOL
sorts).
%
For example, \texttt{p} and \texttt{q}, which are Boolean constants (MSFOL
constants can be seen as free variables when they are not pre-defined, which is
the case for \texttt{p} and \texttt{q} but not for example for \texttt{false}),
are declared as:

\begin{minted}{lean}
def p: term := const 1000 boolSort
def q: term := const 1001 boolSort
\end{minted}

The identifiers \texttt{1000} and \texttt{1001} are arbitrary, with only the
requirement that they are unique.

With these terms and using the reserved symbols \texttt{implies} and
\texttt{not} from the deep embedding corresponding to the MSFOL symbols
$\rightarrow$ and $\neg$, respectively, we encode the formula used in the
query:

\begin{minted}{lean}
def notModusPonens: term :=
  not (implies p (implies (implies p q) q))
\end{minted}

Finally, the proof from Figure~\ref{fig:cvc5-proof} is encoded as

\begin{minted}{lean}
theorem th0 : thHolds notModusPonens -> thHolds bot :=
  fun lean_a0 =>
    have lean_s0 := notImplies2 lean_a0
    have lean_s1 := notImplies1 lean_s0
    have lean_s2 := impliesElim lean_s1
    have lean_s4 := notImplies1 lean_a0
    have lean_s6 := R1 (conjunction lean_s2 lean_s4)
    have lean_s9 := notImplies2 lean_s0
    contradiction (conjunction lean_s9 lean_s6)
\end{minted}

where \texttt{thHolds} is a Lean axiom that lifts a \texttt{term} into a \texttt{Prop} (which is the Lean type that represents propositions), asserting that it is a valid term. All the inference rules are encoded as axioms that have the type \texttt{thHolds $\alpha$ -> thHolds $\beta$}, for some terms $\alpha$ and $\beta$. For instance, \texttt{notImplies1} is written in the following form:
\begin{minted}{lean}
axiom notImplies1 : forall {t1 t2 : term},
  thHolds (not (implies t1 t2)) -> thHolds t1
\end{minted}

By applying the rules in the sequence generated by cvc5, we can derive a term
that has type \texttt{thHolds notModusPonens -> thHolds bot}, where \texttt{bot}
is the encoding of MSFOL's \texttt{false}.
%
The existence of this term shows that, assuming the rules used by cvc5, the term \texttt{notModusPonens} is equivalent to \texttt{bot}, which is the same as to say that it is unsatisfiable. This way, we have encoded the proof found by cvc5 inside Lean.

    \section{First Approach: Certified Transformations}
    Given this framework for encoding MSFOL in Lean, our goal is to
modify its architecture so that the validity of the rules can be
checked by the kernel, and also, in a way that enables the proofs to refer
to native Lean values, as opposed to only values encoded by the
\texttt{term} type. In this section we describe our first attempt
to achieve this goal, which was using the certified transformations
approach, as described in the introduction.

\subsection{The Boolean Fragment}

Initially, we will limit ourselves to the fragment of MSFOL that
deals with Boolean values\footnote{The Boolean fragment of the
\emph{Core} theory in SMT-Lib, as defined in
\url{http://smtlib.cs.uiowa.edu/theories-Core.shtml}}. Once
we show how to lift this fragment, we will present a generalization
of our definitions that could potentially serve as a basis to lift
any theory from MSFOL.\
%
The first step is to define a function to map values from the \texttt{term} type
to the corresponding native value.

While designing this function, we had a
choice on which type from Lean we would use as a counterpart of \texttt{term}. The two
most suitable alternatives were the \texttt{Prop} and \texttt{Bool} types. The former, as
previously explained, is the type used to model all propositions in the language, while
the latter is the usual type of booleans inhabited by only two values: \texttt{true} and
\texttt{false}. The \texttt{Bool} type has the advantage of potentially not requiring the
\texttt{Classical} module, as it is possible to prove classical statements over them. On
the other hand, propositions in Lean are stated in terms of \texttt{Prop}s.
\tom{I'm not sure about the next statement} Although it is
possible to state them in terms of \texttt{Bool}s, it is not usual. Given that one of the
end goals of our project is to serve as part of a Lean hammer, it is more appropriate
to adopt the most commonly format used. With this in mind, we chose to use \texttt{Prop}.


Note that the term we will evaluate can contain free variables. For those, we will need an auxiliary interpretation function assigning concrete values to them.
Free variables are
identified by a \texttt{Nat} (the built-in Lean type for natural numbers), therefore, we can
represent this information as a function from \texttt{Nat} to
\texttt{Prop}:

\begin{minted}{lean}
  def Interpretation := Nat -> Prop
\end{minted}

With this definition, we can define our evaluation function:

\begin{minted}{lean}
  def evalTerm (I : Interpretation) (t : term) : Prop :=
    match t with
    | term.const   i  _  => I i
    | term.not     t1    => Not (evalTerm I t1)
    | term.and     t1 t2 => And (evalTerm I t1) (evalTerm I t2)
    | term.or      t1 t2 => Or (evalTerm I t1) (evalTerm I t2)
    | term.implies t1 t2 => (evalTerm I t1) -> (evalTerm I t2)
    | term.eq      t1 t2 => (evalTerm I t1) = (evalTerm I t2)
    | term.bot           => False
    | term.top           => True
    | _                  => False
\end{minted}

This function is matching each pattern for a \texttt{term} with the corresponding built-in operation over \texttt{Prop}, and using recursive calls of itself as arguments. If we find a \texttt{term} that is not in the fragment we are currently supporting we just return \texttt{False}. This could potentially lead to consistency issues if the input formula involved other fragments apart from Boolean. As we are limiting ourselves to this fragment, we will ignore this problem for now.

Notice how the \texttt{Interpretation} type we introduced, as well as the evaluation function, match the notions of interpretation and evaluation introduced in Section~\ref{sec:msfolHere}. Indeed, we can now define what it means for an interpretation to satisfy a \texttt{term} and what it means to be unsatisfiable:

\begin{minted}{lean}
  def satisfies (I : Interpretation) (t : term) : Prop :=
    evalTerm I t = True
  def unsatisfiable (t : term) : Prop :=
    ∀ (I : Interpretation), ¬ satisfies I t
\end{minted}

One important concept we need to define is what it means for a \texttt{term}
to follow logically from another, which is the primary relationship modeled
by the axioms presented previously.
%
If, for any fixed interpretation, the evaluation of a given \texttt{term} being true
implies in the evaluation of another \texttt{term} being true, then we can always
conclude the second one from the first. The following definition states this
relationship in Lean:

\begin{minted}{lean}
  def impliesIn (t1 t2 : term) : Prop :=
    ∀ (I : Interpretation),
      satisfies I t1 -> satisfies I t2
\end{minted}

Notice that we needed to use the same interpretation for both terms.
%
The application \texttt{impliesIn t1 t2} gives a precise meaning to \texttt{thHolds t1 -> thHolds t2} in terms of the logic used by Lean.

Since we are interested in proving the unsatisfiability of terms, we will always try to prove a goal of the form \texttt{impliesIn t bot}, for some term \texttt{t}. This would imply that for any interpretation \texttt{I}, we have \texttt{(evalTerm I t = True) -> False}, provided that there is no environment that validates the interpretation of \texttt{bot}. Note that this is equivalent to \texttt{unsatisfiable t}, given our previous definition of \texttt{unsatisfiable}.
With the above we can rephrase and prove the rules from the deep embedding using
\texttt{impliesIn}.
%
For instance, the representation of \texttt{notImplies1} show in Section~\ref{sec:gen-scripts} becomes:

\begin{minted}{lean}
  theorem notImplies1 : ∀ {t1 t2 : term},
      impliesIn (not (implies t1 t2)) t1
\end{minted}

We have proved some of the theorems used in the boolean fragment\footnote{Our proofs can be found at \url{https://github.com/tomaz1502/signatures/blob/smallCheckers/Cdclt/Lift/Other/PropsExample.lean}.}. Their proofs
were straightforward using classical reasoning.

Finally, we can state the theorem from Section~\ref{sec:gen-scripts} as
\texttt{impliesIn notModusPonens bot}, and prove it using almost\footnote{while the original proof used the most general form of resolution, we have restricted ourselves to prove the specific version of resolution that was applied in this proof.} the same
(rephrased) rules in the same order, which already achieves the goal
of checking the proof using Lean's kernel.

In order to apply the generated proof in terms native to the ITP, we need
to prove the following auxiliary theorem:

\begin{minted}{lean}
  theorem notFollowsBot : ∀ {t : term},
    impliesIn (not t) bot → ∀ {I : Interpretation}, evalTerm I t = True
\end{minted}

By applying it on the theorem generated by cvc5, we derive that, for any
interpretation \texttt{I}, \texttt{evalTerm I modusPonensEmbed} is equal
to \texttt{True}:

\begin{minted}{lean}
theorem modusPonensEqTrue: ∀ {I: Interpretation},
    evalTerm I modusPonensEmbed = True :=
  notFollowsBot cvc5_th0
\end{minted}

Now we can define the theorem corresponding to \texttt{cvc5\_th0} using
\texttt{Prop}s and prove it by applying \texttt{modusPonensEqTrue} to
the appropriate interpretation:

\begin{minted}{lean}
def modusPonens (P Q : Prop) : Prop := P → (P → Q) → Q

theorem modusPonensCorrect: ∀ (P Q: Prop), (modusPonens P Q) = True := by
  intros P Q
  exact @modusPonensEqTrue (fun id => if id == 1000 then P else Q)
\end{minted}

where the symbol \texttt{@} is used to make explicit all parameters
in the function after it. Using this interpretation, we indicate
that the term \texttt{p} in \texttt{modusPonensEmbed}
will be matched with the prop \texttt{P} (since the identifier of \texttt{p}
is 1000) and the term \texttt{q} will be matched with the prop \texttt{Q}.
The checker can now compute our evaluation function and
match its return value with \texttt{modusPonens P Q}, thus proving the theorem.

Therefore, we have shown how to lift the proofs produced by cvc5 into proofs
that refer directly to native Lean terms. In order to fully automate this
process, we would have to extend cvc5's module for printing proofs. It would
have to also print, for a given query, the theorem (and its proof, which is always
\texttt{notFollowsBot cvc5\_th0}) corresponding to \texttt{modusPonensEqTrue} for that query,
the representation of the query as a Lean term (which in this example was the term \texttt{modusPonens})
and the theorem that proves that the representation of the query is correct by instantiating the interpretation
properly (corresponding to \texttt{modusPonensCorrect}).
Since we changed our approach, this extension was not realized.

\subsection{Supporting Other Theories}

Supporting more theories from MSFOL requires extending the function \texttt{evalTerm},
as well as the \texttt{Interpretation} type we defined, to be able to return values
of multiple distinct types. One type safe way to achieve this in a language with
dependent types is through a \textit{sigma type}. A sigma type is a pair, in which
the type of the second element depends on the value of the first element. If
\texttt{T} is a type and \texttt{U} is a constructor with type
\texttt{T -> Type}, then \texttt{@Sigma T U} is the type
of pairs \texttt{⟨t, u⟩} such that \texttt{t} has type \texttt{T} and
\texttt{u} has type \texttt{U t}. Note that the first parameter of \texttt{Sigma}
can be inferred from the second, so it is given implicitly.

Let us define a function \texttt{evalSort} that maps the type \texttt{sort}
(corresponding to MSFOL's sorts in the deep embedding) to native Lean types
(which are represented by \texttt{Type}):

\begin{minted}{lean}
  def evalSort : sort -> Type := fun s =>
    match s with
    | arrow s1 s2 => evalSort s1 -> evalSort s2
    | boolSort => Prop
    | intSort => Int
    | _ => Prop
\end{minted}

We have matched the \texttt{arrow} sort with the arrow type used to build the type of
functions, \texttt{boolSort} with \texttt{Prop} and \texttt{intSort} with \texttt{Int}.
For giving support for further theories we have to extend this match statement,
matching the corresponding \texttt{sort} with a suitable type. Since one of the goals
of the project is to be a hammer, it is crucial to choose, for a given sort,
a corresponding type that is most commonly employed by Lean users, and also
has the same properties as the sort.

Now we can reformulate the \texttt{Interpretation} type, in a way that
it supports any type that is also supported by \texttt{evalSort}:

\begin{minted}{lean}
  def Interpretation := Nat → @Sigma sort evalSort
\end{minted}

Given an identifier, an interpretation must return a pair containing its sort
and its value. Note that, with this modification, the interpretation printed
by cvc5 would also need to print the sort of each term. For instance,
the interpretation used in the theorem \texttt{modusPonensCorrect} would have
to be rewritten as:

\begin{minted}{lean}
  def I : Interpretation := fun id =>
    if id == 1000 then
      ⟨ boolSort, P ⟩
    else ⟨ boolSort, Q ⟩
\end{minted}

We will also use a sigma type for defining the new version of \texttt{evalTerm}.
%
Since we are now supporting multiple types, we have to consider what to return
when the input \texttt{term} is ill-typed. One possibility could be to map these
terms to \texttt{False}, as we were doing with \texttt{term}s that were not supported
in the previous version of \texttt{evalTerm}. However, this approach would introduce
a logical inconsistency that would pose challenges to prove some of the rules.
For instance, one rule used by cvc5 is the elimination of double negation:

\begin{minted}{lean}
  theorem notNotElim : ∀ {t : term},
      impliesIn (not (not t)) t
\end{minted}

In order to prove it, we have to consider every possible pattern for \texttt{t}.
If \texttt{t} is not a valid boolean expression, then our evaluation function would
return \texttt{False} for the term \texttt{not t}, which would then force the
evaluation of \texttt{not (not t)} to return \texttt{True}. Since the premiss
is valid in this case, we would have to prove that the conclusion is also valid,
but the conclusion is not a boolean expression. Therefore, the only way to prove it
would be to change our predicate \texttt{satisfies} to accept terms that are not booleans.

Instead of following this approach, we decided to change the evaluation function to just not return
any value if the input term is ill-typed. The polymorphic \texttt{Option} type is used in Lean to
indicate the possible absence of a value, which is represented by \texttt{none}, one of its
constructors. The other constructor, \texttt{some}, receives, as a parameter, a single value
of the type that is used as a parameter to \texttt{Option}.
%
The following snippet shows the reformulated version of \texttt{evalTerm}.
We do not show the complete pattern matching for brevity, but all the other
patterns are implemented using the same structure:

\begin{minted}{lean}
  def evalTerm (I : Interpretation) (t : term) :
      Option (@Sigma sort interpSort) :=
    match t with
    | term.const   i  s  =>
      let ⟨ s', value ⟩ := I i
      if s' == s then some ⟨ s', value ⟩ else none
    | term.and     t1 t2 =>
      match evalTerm I t1, evalTerm I t2 with
      | some ⟨ boolSort, p1 ⟩, some ⟨ boolSort , p2 ⟩ =>
          some ⟨ boolSort, And p1 p2 ⟩
      | _,_ => none
    | _ => none
\end{minted}

We have also reformulated our \texttt{satisfies} predicate, in a way that
it also rejects any term whose evaluation is not a \texttt{Prop}:

\begin{minted}{lean}
  def satisfies (I : Interpretation) (t : term) : Prop :=
    match evalTerm I t with
    | some ⟨ boolSort, p ⟩ => p = True
    | _ => False
\end{minted}

The predicate \texttt{impliesIn} did not require any modification. With these
rephrased definitions we could define and prove new versions of
the theorems regarding the boolean fragment, certifying the validity of the
cvc5 rules over the reformulated version of our framework.

The next step we took was to try to state and prove theorems corresponding to the
axioms from the theory of equality and uninterpreted functions, which
was presented in Section~\ref{sec:euf}. Consider the \texttt{refl} axiom:

\begin{minted}{lean}
  theorem refl : ∀ {I : Interpretation} {t : term}, satisfies I (eq t t)
\end{minted}

If \texttt{t} is ill-typed, then the statement does not hold. This was not
a problem before because all previous axioms were implications in which
the term in the conclusion was a subexpression of the premiss. Therefore,
if the conclusion was ill-typed, the premiss was also, necessarily, ill-typed,
which made the implication true. In the case of \texttt{refl} and of
any other possible theorems that do not have premisses, we have to restrict it
to only refer to well-typed terms in order to make their statement true.

For that purpose, we have defined a new function \texttt{inferSort} that infers the sort of a term
or returns \texttt{none} if it is ill-typed.
This new function is essentially identical to
\texttt{evalTerm}, except that it only computes the first element of the pair.
Also, since the sort of a term is independent of the interpretation we use
to evaluate it, we do not need an interpretation as a parameter in this function.
Now we can make the statement in the theorem \texttt{refl} true by adding the hypothesis
\texttt{isSome (inferSort t) = true}, where \texttt{isSome} is:

\begin{minted}{lean}
  def isSome (opt : Option sort) : Bool :=
    match opt with
    | some _ => true
    | none   => false
\end{minted}

The introduction of this hypothesis creates a new requirement for applying
the theorem in a proof, that is, we have to provide a proof that \texttt{t} is well-typed.
If it is well-typed (which will always be the case as long
as there is no bug in the SMT solver), then Lean's kernel can evaluate
both functions \texttt{inferSort} and \texttt{isSome} and obtain \texttt{true},
and the proof that the term is well-typed follows by reflexivity. Deriving facts
from the evaluation of functions is a proof technique known as \textit{proof by reflection}.
This kind of proof transfer the cost of \textit{checking} the proofs to a cost of
normalizing terms. Unfortunately, as pointed out by~\cite{ringLean}, Lean's evaluator
is not optimized for performance, therefore, the necessity of proving that certain term are
well-typed would probably reduce the efficiency of our tool.

Another downside of this approach is the necessity of providing explicit proofs for the most
general case of all rules. Some of the rules, such as resolution and
\textit{factor} (which takes as premiss a clause and have as a conclusion the same clause, removing
all formulas that are duplicated) appear to lack any proofs that are not highly challenging to derive.
This difficulty is also recognized in~\cite[6]{snipe}:
\begin{quote}
  ``Indeed, the former [certified transformations] require we work only with the reified [meaning deeply embedded] syntax of the
  terms of CIC and even for a simple transformation, the proof of soundess is hard (thousands lines of code).''
\end{quote}

As we will see later, the second approach does not require to build all the proofs \textit{explicitly}.
For those reasons, we decided to change our approach.

% According to~\cite[6]{snipe},

% From leanRingCertifying.pdf (paper about extending ring tactic with certifying
% transformations):
% % \cite{ringLean}
% The ring exp tactic does not use reflection but directly constructs proof
% terms to be type checked by Lean’s kernel, as is typical for tactics in mathlib [10].
% Reflective tactics avoid the construction and checking of a large proof term by
% performing most computation during proof checking, running a verified pro-
% gram [2]. If the proof checker performs efficient reduction, this results in a sig-
% nificant speed-up of the tactic, at the same time as providing more correctness
% guarantees. Unfortunately, the advantages of reflection do not translate directly
% to Lean. Tactic execution in Lean occurs within a fast interpreter, while the
% kernel used in proof checking is designed for simplicity instead of efficient reduc-
% tion [3]. Achieving an acceptable speed for ring exp requires other approaches
% to the benefits that reflection brings automatically.

% \tom{evalterm will have to be evaluated during the checking of the proof produced by cvc5
%   thats a performance issue = references}

% \tom{factor and resolution and permutateOr are exceedingly hard to be proved. From snipe:
% In the work presented in this article, we follow the paradigm of certifying transformations.
% Indeed, the former require we work only with the reified syntax of the terms of CIC and even for
% a simple transformation, the proof of soundess is hard (thousands lines of code).}

% \tom{Haniel said that Chantal said that it is hard to maintain the verification, take a look at her papers}

    \section{Second Approach: Certifying Transformations}
    In this chapter we describe our main contribution, which is a module
for lifting the scripts presented in Section~\ref{sec:gen-scripts} into
Lean proofs, based on the certifying transformations approach.
%
The specific manner in which we applied
this concept in our context was to develop a set of tactics matching
the rules present in cvc5's proof calculus.
A cvc5 proof is then a sequence of applications of those tactics.
When we ask Lean's kernel to check the proof script, it will execute the tactics
one by one. Each one of them inspects the current context
and produces a proof term corresponding to a proof of a specific
case of the rule represented by that tactic.
The checker then verifies the correctness of each proof, closing the original goal only
if all checks were succesful.

There are two options for implementing these tactics: the first one is to generate
a value of type \texttt{Syntax} (that is, a piece of Lean's code) that proves
the theorem in question and invoke \texttt{evalTactic} using this value;
the second option is to craft an \texttt{Expr} that corresponds to a value
of the appropriate type. In general,
it is simpler to generate \texttt{Syntax}, as the elaborator will fill in
some details for us, such as the implicit arguments for functions. On the
other hand, we can skip calls to the elaborator if we produce an \texttt{Expr}
instead, which will potentially lead to a gain in performance. With this in mind,
we decided to employ the second option.


Another important difference from the certified transformations approach is that we do not use the \texttt{term}
type anymore. The main reason for mapping MSFOL formulas to this inductive type
instead of native Lean expressions is the flexibility achieved by this representation.
While it is straightforward to define a function that inspects and manipulates the structure of
a \texttt{term} by pattern matching, there is no way to do the same for certain Lean types (\texttt{Prop}, for
instance) without recurring to metaprogramming. As we have shown in Section~\ref{sec:metaLean},
the metaprogramming context grants us access to the internal representation of any expression
through the \texttt{Expr} type, which can be inspected and manipulated in the same way
as \texttt{term}. Since the framework for writing tactics is based on metaprogramming, we
do not need to rely on the flexibility of the \texttt{term} type. Therefore, we
have decided to not use the deep embedding anymore, and translate MSFOL formulas directly
to Lean expressions.

The snippet in Figure~\ref{notModusPonens2} shows an example of proof in the new format.
It corresponds to the proof that cvc5 produces to the SMT-Lib query presented
in Figure~\ref{negModusPonens}. A considerable portion of the rules (including
every rule regarding CNF transformation) can be easily proved using classical
reasoning. Instead of mapping such rules to tactics, we just proved them as theorems.
Those theorems will not be presented here. For a complete overview of the rules, their
statement and whether they were implemented with a tactic or a
theorem, refer to Table~\ref{tab:rules}.
All the rules present in the proof in Figure~\ref{notModusPonens2} were mapped to
theorems except for resolution (line 8), which was mapped to a tactic that we will
present.


\begin{figure}[t]
\begin{minted}[linenos]{lean}
  theorem cvc5_th0 {P Q : Prop} : (Not (P → ((P → Q) → Q))) → False :=
    fun lean_a0 : (Not (P → ((P → Q) → Q))) => by
      have lean_s0 : (Not ((P → Q) → Q)) := notImplies2 lean_a0
      have lean_s1 : (P → Q)             := notImplies1 lean_s0
      have lean_s2 : (Or (Not P) Q)      := impliesElim lean_s1
      have lean_s3 : P                   := notImplies1 lean_a0
      have lean_s4 : Q                   :=
        by R2 lean_s2, lean_s3, P, [1, 0]
      have lean_s5 : (Not ((P → Q) → Q)) := notImplies2 lean_a0
      have lean_s6 : (Not Q)             := notImplies2 lean_s5
      exact (show False from contradiction lean_s4 lean_s6)
\end{minted}
\caption{Proof script using certifying transformations.}\label{notModusPonens2}
\end{figure}

\paragraph{Representation of clauses.} Internally, clauses are represented by cvc5 as
a lists of terms. A list of terms can correspond to many distinct
clauses, depending on how you parenthesize them. Implicitly, cvc5 is using the
fact that disjunction is associative, which implies that all these clauses
are equivalent, therefore they do not need to be differentiated inside the calculus.
When these lists of terms are sent to Lean, we are forced to chose a way to parenthesize
those terms and build our proofs taking into account this format.
We chose to parenthesize them in a right-associative way.
Therefore, if cvc5 is
representing some clause as the list $[t_{1}, t_{2}, t_{3}, t_{4}]$ we will represent it in Lean as
\texttt{t₁' ∨ (t₂' ∨ (t₃' ∨ t₄'))}, where \texttt{tᵢ'} is the representation of $t_{i}$.
Since the built-in operator \texttt{∨} is right-associative, we can omit the
parenthesis. Throughout the implementation of the tactics, we always assume
that a clause that is received as premise comes parenthesized in this way,
and every clause that is proved by a tactic also has this format.

In the rest of this chapter we will give an overview of the implementation of some of the
tactics. We present their statements with the same format used in cvc5's
documentation\footnote{The documentation of cvc5's rules can be found at:
  \url{https://cvc5.github.io/docs/cvc5-1.0.2/proofs/proof_rules.html}}, that is,
for a tactic that has a conclusion $\psi$, premisses $\psi_{1}, \cdots, \psi_{n}$ and
parameters $t_{1}, \cdots, t_{n}$, we write:
\[
  \infer[]{\psi}{\psi_{1}, \cdots, \psi_{n} \mid t_{1}, \cdots, t_{n}}
\]

\section{Auxiliary Tactics}

This first class of tactics does not correspond to cvc5's rules. They were implemented
to facilitate the implementation of the main tactics.



\subsection{Parenthesizing prefixes of clauses}

Given a number \texttt{i} and a proof \texttt{pf}
of the validity of a clause, the application \texttt{groupClausePrefix pf, i} proves the same clause, with the
first $i$ propositions parenthesized. The disjunction of the first $i$ propositions is a single element of the output clause.  The precise statement of this rule
is the following:

\[
  \infer[]{(P_{1} \vee \cdots \vee P_{i}) \vee P_{i + 1} \vee \cdots \vee P_{n}}
    {P_{1} \vee \cdots \vee P_{n} \mid i}
\]

This tactic employs the following two theorems, which were easily proven:

\begin{itemize}
  \item \mintinline{lean}{orAssoc {A B C : Prop} : A ∨ B ∨ C → (A ∨ B) ∨ C}
  \item \mintinline{lean}{congOrLeft {A B C : Prop} (hyp : A → B) : C ∨ A → C ∨ B}
\end{itemize}

% Also, it will make use of the \texttt{apply} tactic. If the current goal is \texttt{Q}, and
% we have a proof \texttt{h} of \texttt{P → Q}, we can reduce the goal to \texttt{P} with
% \texttt{apply h}.

% Let's instantiate, in the theorem \texttt{orAssoc}, \texttt{A} to $P_{1}$, \texttt{B}
% to $P_{2} \vee \cdots \vee P_{i}$ and \texttt{C} to $P_{i + 1} \vee \cdots \vee P_{n}$. This will yield a proof of
% of $P_{1} \vee (P_{2} \vee \cdots \vee P_{i}) \vee P_{i + 1} \vee \cdots \vee P_{n} \rightarrow (P_{1} \vee \dots \vee P_{i}) \vee P_{i + 1} \vee \cdots \vee P_{n}$. Since
% the proof script is annotated with the type it expects from each tactic application,
% we can assume that the current goal is $(P_{1} \vee \cdots \vee P_{i}) \vee P_{i + 1} \vee \cdots \vee P_{n}$. Therefore, we can change the goal to $P_{1} \vee (P_{2} \vee \cdots \vee P_{i}) \vee P_{i + 1} \vee \cdots \vee P_{n}$ with \texttt{apply orAssoc} (Lean's kernel can infer what are the implicit arguments in this case). Now we would like to apply \texttt{orAssoc} again to remove $P_{2}$ from the parenthesized group. We cannot do it directly, due to the term $P_{1}$ to the left of the parenthesis. We solved this
% problem by using \texttt{congOrLeft}, with \texttt{C} instantiated to $P_{1}$ and
% \texttt{hyp} instantiated to \texttt{orAssoc}, which yields a proof of the correct
% statement. Therefore, we can change the goal to $P_{1} \vee P_{2} \vee (P_{3} \vee \cdots \vee P_{i}) \vee \cdots \vee P_{n}$ with \texttt{apply (congOrLeft orAssoc)}.

First, let's build a proof term for $P_{1} \vee \cdots \vee (P_{i - 1} \vee P_{i}) \vee \cdots \vee P_{n}$. We instantiate, in \texttt{orAssoc}, the parameter \texttt{A} to $P_{i - 1}$, \texttt{B} to $P_{i}$ and \texttt{C} to $P_{i + 1} \vee \cdots \vee P_{n}$, obtaining the term:

\begin{center}
    $f_{1}: P_{i - 1} \vee P_{i} \vee \cdots \vee P_{n}   \rightarrow (P_{i - 1} \vee P_{i}) \vee \cdots \vee P_{n} $
\end{center}

Then, we apply \texttt{congOrLeft} to $f_{1}$ instantiating \texttt{C} to $P_{i - 2}$, which produces:

\begin{center}
    $f_{2}: P_{i - 2} \vee P_{i - 1} \vee P_{i} \vee \cdots \vee P_{n}   \rightarrow P_{i - 2} \vee (P_{i - 1} \vee P_{i}) \vee \cdots \vee P_{n} $
\end{center}

We repeat this process until we get the term $f_{i - 1}$ of type $P_{1} \vee \cdots \vee P_{n} \rightarrow P_{1} \vee \cdots \vee (P_{i - 1} \vee P_{i}) \vee \cdots \vee P_{n}$. Applying $f_{i - 1}$ to the original hypothesis \texttt{h} yields a term \texttt{h'} with type $P_{1} \vee \cdots \vee (P_{i - 1} \vee P_{i}) \vee \cdots \vee P_{n}$.

A clause is composed by any kind of propositions, including disjunctions.
Therefore, the term $P_{1} \vee \cdots \vee (P_{i - 1} \vee P_{i}) \vee \cdots \vee P_{n}$ is a new clause
with $n - 1$ propositions, in which the $(i - 1)$-th proposition is the disjunction
$P_{i - 1} \vee P_{i}$. This means that our original problem can be reduced to
group the first $i - 1$ terms of this clause.
With this in mind, we repeat the process of composing \texttt{congOrLeft} with
\texttt{orAssoc} and apply the result to \texttt{pf'}, obtaining
a new term of type $P_{1} \vee \cdots \vee (P_{i - 2} \vee P_{i - 1} \vee P_{i}) \vee \cdots \vee P_{n}$.
We keep repeating this process until the whole prefix is grouped.


\begin{figure}[t]
\begin{minted}[linenos]{lean}
def groupPrefixCore (pf : Expr) (i : Nat) : MetaM Expr := do
  let clause  : Expr      ← inferType pf
  let props   : List Expr ← collectPropsInClause clause
  if i > 0 && i < List.length props then
    let lemmas : List Expr ← groupPrefixLemmas props (i - 1)
    let answer : Expr :=
      List.foldl (fun acc lem => Expr.app lem acc) pf lemmas
    return answer
  else throwError
    "[groupClausePrefix]: invalid prefix length"
\end{minted}
\caption{Implementation of the tactic GroupClausePrefix}\label{groupClause}
\end{figure}

The snippet in Figure~\ref{groupClause} and shows the
implementation of the core functionality of this tactic.
The parameters \texttt{pf} and \texttt{i} of the function \texttt{groupPrefixCore}
represent, respectively, the proof of the validity of the original clause
and the length of the prefix that should be grouped. In line 2, we obtain
an \texttt{Expr} corresponding to the original clause by inspecting the type
of \texttt{pf}, with the built-in routine \texttt{inferType}. Then, in line 3,
we use the function \texttt{collectPropsInClause} (defined by us) to
extract a \texttt{List} with each one of the propositions in the clause.
Next, we check if the prefix length is valid. If it is, we apply our function
\texttt{groupPrefixLemmas} to obtain a list of expressions. Each element in this list
is a composition of \texttt{congOrLeft} and \texttt{orAssoc} described before.
One important difficulty in the implementation of this function is that implicit parameters cannot always be automatically computed in the \texttt{Expr} level, so
this function also computes some of the implicit arguments that have to be
passed to \texttt{congOrLeft} and \texttt{orAssoc}. Lean has a built-in
functionality to automatically infer part of those arguments, but a significant
amount of them have to be manually constructed.
Finally, we use \texttt{foldl}
in line 7 to apply each one of the lemmas to \texttt{pf}, accumulating the results.

In order to use \texttt{groupPrefixCore} as a tactic, we implement
a function of type \texttt{Syntax → TacticM Unit}, which parses the \texttt{Syntax}
to obtain \texttt{pf} and \texttt{i} and use the \texttt{Expr} generated by
\texttt{groupPrefixCore} to close the current goal.

The following snippet shows an example of usage of this tactic:

\newpage

\begin{minted}{lean}
  theorem group3 : A ∨ B ∨ C ∨ D ∨ E → (A ∨ B ∨ C) ∨ D ∨ E := by
    intro h
    groupClausePrefix h, 3
\end{minted}

We can use the \texttt{\#print} command to inspect the proof term generated by the tactic, and
verify that it is, indeed, the one we described:

\begin{minted}{lean}
  #print group3 -- fun {A B C D E} h => orAssoc (congOrLeft orAssoc h)
\end{minted}
% \tom{proof term size is $\mathcal{O}(i^{2})$. Maybe say a few words about this?}

\subsection*{Moving terms inside clauses}

Given a proof \texttt{pf} of the validity of a clause and an index \texttt{i} of a proposition to be moved, the \texttt{pull} tactic should produce a proof term for the same clause,
with $i$-th term moved to the first position. This tactic also facilitates the implementation
of the other tactics and does not exist in cvc5's proof calculus. The precise
statement of this rule is the following:

\[
  \infer[]{P_{i} \vee P_{1} \vee \cdots \vee P_{i - 1} \vee P_{i + 1} \vee \cdots \vee P_{n}}{P_{1} \vee \cdots \vee P_{n}  \mid i, s}
\]


We need three theorems to implement \texttt{pull}, which are easy to prove:

\begin{itemize}
  \item \mintinline{lean}{orComm {A B C : Prop} : A ∨ B → B ∨ A}
  \item \mintinline{lean}{congOrRight {A B C : Prop} (hyp : A → B) : A ∨ C → B ∨ C}
  \item \mintinline{lean}{orAssocConv {A B C : Prop} : (A ∨ B) ∨ C → A ∨ B ∨ C}
\end{itemize}

To build the required proof term, let's first apply our tactic \texttt{groupClausePrefix} at \texttt{pf}, grouping the prefix of length $i - 1$.
We cannot use the tactic itself, as we are working on the context of \texttt{MetaM}, but we can invoke directly the function \texttt{groupPrefixCore}
to produce the required term. We will obtain the following term:

\begin{center}
  $pf_{1} : (P_{1} \vee \cdots \vee P_{i - 1}) \vee P_{i} \vee \cdots \vee P_{n}$
\end{center}

Then, we use again our function \texttt{groupPrefixCore} in \texttt{pf₁}, grouping the prefix of length 2. This provides the term:

\begin{center}
  $pf_{2} : ((P_{1} \vee \cdots \vee P_{i - 1}) \vee P_{i}) \vee P_{i + 1} \vee \cdots \vee P_{n}$
\end{center}

Next, we will define a new term \texttt{pf₃} as \texttt{congOrRight orComm pf₂}. Given the appropriate instantiation for the implicit parameters,
this application flips the position of $P_{1} \vee \cdots \vee P_{i - 1}$ and $P_{i}$, while not modifying the suffix $P_{i + 1} \vee \cdots \vee P_{n}$.
Therefore, \texttt{pf₃} has type $(P_{i} \vee P_{1} \vee \cdots \vee P_{i - 1}) \vee P_{i + 1} \vee \cdots \vee P_{n}$ (we do not need to parenthesize
$P_{1} \vee \cdots \vee P_{i - 1}$ as $\vee$ is right-associative). Finally we use another tactic we implemented, \texttt{ungroupClausePrefix},
to remove the parenthesis around $P_{i} \vee P_{1} \vee \cdots \vee P_{i - 1}$, concluding our goal.
This tactic is similar to \texttt{groupClausePrefix}. The
difference is that it substitutes the theorem \texttt{orAssoc} by \texttt{orAssocConv} and it applies the theorems in reverse order, i.e.\ it
starts with an application of \texttt{orAssocConv}, then it applies \texttt{congOrLeft orAssocConv} and so on.

\paragraph{Disambiguating clauses.} The difference between the representation of
clauses in cvc5 and in Lean leads to an ambiguity in applications of this tactic.
While there is no distinction in Lean between, for instance, the clauses $A \vee B \vee C \vee D \vee E$
and $A \vee B \vee C \vee (D \vee E)$, this distinction can be made inside the SMT solver. Indeed, if
we have a term \texttt{pf} in Lean of type $A \vee B \vee C \vee D \vee E$ and cvc5 prints the tactic
application \texttt{pull pf, 4}, there is no way to know, inside Lean, if it is expecting to receive
a proof of $D \vee A \vee B \vee C \vee E$ or $(D \vee E) \vee A \vee B \vee C$. To address this issue,
we adapted cvc5 to print an extra parameter \texttt{s} to this tactic (and to others that suffered from the same problem),
which is a natural number corresponding to the index of the last proposition in the clause from the
point of view of the SMT solver. Therefore, for the example we just described, it would print
\texttt{pull pf, 4, 4} to indicate that it wants a proof for $(D \vee E) \vee A \vee B \vee C$ and
\texttt{pull pf, 4, 5} to indicate that it wants a proof for $E \vee A \vee B \vee C \vee D$.

\section{Boolean reasoning}

We now present the tactics regarding Boolean reasoning. One of them corresponds to
the resolution rule, which as we have shown in Section~\ref{sec:pcBool}, is the main
building block for proofs in Propositional Logic. The other two, \textit{factor} and
\textit{permutateClause}, are used to justify implicit steps that SAT solvers
perform while solving formulas through resolution.

\subsection*{Reordering clauses}

Given the proof \texttt{pf} of a clause, a permutation \texttt{perm} and the index \texttt{s} of the last
proposition in the clause, the application \texttt{permutateClause pf, perm, s} provides a proof for the same clause, with
the propositions permutated according to \texttt{perm}.
This is the first tactic that matches a rule used by cvc5\footnote{The documentation of this rule can be found at \url{https://cvc5.github.io/docs/cvc5-1.0.2/proofs/proof\_rules.html\#\_CPPv4N4cvc58internal6PfRule10REORDERINGE}}. The precise statement
of this rule is the following:

\[
  \infer[]{P_{perm(1)} \vee \cdots \vee P_{perm(n)}}{P_{1} \vee \cdots \vee P_{n} \mid perm, s}
\]


Since we have \texttt{pull}, we can write a tactic matching
this rule using a relatively straightforward approach. We simply iterate through the permutation in
reverse order, and, for each index \texttt{i} we go through, we run the \texttt{pull} tactic
with \texttt{i} as the argument. With this strategy, the last term we will bring to the first position is
$P_{perm(1)}$, therefore, the final clause we will produce will have the correct element in the first position.
Similarly, the second to last element we will bring to the first
position is $P_{perm(2)}$. After pulling it, we will bring another term to the first position ($P_{perm(1)}$), which will make $P_{perm(2)}$
end up in the second position, as required. By extending this reasoning to all the terms in the clause,
we can conclude that each one of them will be placed in the right position. Notice that the index \texttt{s} of
the last proposition does \textit{not} change during this process, so we can always use \texttt{s} as the argument
required by \texttt{pull}.

\begin{figure}[t]
\begin{minted}[linenos]{lean}
  def pullProps (props : List Expr) (acc : Expr) (s : Nat) :
      MetaM Expr :=
    match props with
      | [] => return acc
      | e::es => do
          let pulled ← pullCore e acc s
          pullProps es pulled s

  def permutateClauseCore (pf : Expr) (perm : List Nat)
      (s : Nat) : MetaM Expr := do
    let clause : Expr     ← inferType pf
    let props : List Expr ← collectPropsInClause' clause s
    let permutatedProps   := permutateList props (List.reverse perm)
    pullProps permutatedProps pf s
\end{minted}
\caption{Implementation of the tactic permutateClause.}\label{permClauseImp}
\end{figure}

The code in Figure~\ref{permClauseImp} shows our implementation of this tactic.
The main function is \texttt{permutateClauseCore}. It starts by obtaining a list of
\texttt{Expr} corresponding to the terms in the clause using \texttt{inferType}
and \texttt{collectPropsInClause} (lines 11 and 12), in the same manner as
\texttt{groupPrefixCore}. Next, in line 13, it permutates this list according
to the reverse of the input permutation. Finally, in line 14, it invokes
\texttt{pullProps} with the permutated list, \texttt{pf} and \texttt{s}.
This auxiliary method traverses the input list, applying \texttt{pull} to
each \texttt{Expr} it encounters, accumulating the result.

The following snippet shows an example of usage of this tactic:

\begin{minted}{lean}
  theorem perm1 :
      A ∨ B ∨ C ∨ (D ∨ E ∨ F) → (D ∨ E ∨ F) ∨ B ∨ C ∨ A := by
    intro h
    permutateClause h, [3, 1, 2, 0], 3
\end{minted}

\paragraph{Comparison with the certified approach.} An essential observation is that, while proving the theorem that corresponds to this rule appears to be a
challenging task, the complete implementation of the tactic was notably compact and uncomplicated, consisting of fewer
than 60 lines (excluding the source code of \texttt{pull}).


\subsection*{Resolution}

Given two proofs of clauses, \texttt{pfP} and \texttt{pfQ}, a proposition \texttt{A}
(also known as the \textit{pivot}) such that
\texttt{A} is an element of the first clause and \texttt{¬ A} is an element of the second clause and the indices \texttt{s₁} and \texttt{s₂} of the last propositions in each clause, this tactic produces a proof for the clause composed by the concatenation
of the two, removing the pivot from the first and the negation of the pivot from the second. If there are multiple instances of the pivot in the first clause, it removes
the first one. Similarly, if there are multiple instances of the negation of the pivot
in the second clause, it also removes the first one. The precise statement
is the following:

\[
  \infer[]{P_{1} \vee \cdots \vee P_{i - 1} \vee P_{i + 1} \vee \cdots \vee P_{n} \vee Q_{1} \vee \cdots \vee Q_{j - 1} \vee Q_{j + 1} \vee \cdots \vee Q_{m}}{P_{1} \vee \cdots \vee P_{i - 1} \vee A \vee P_{i + 1} \vee \cdots \vee P_{n}, Q_{1} \vee \cdots \vee Q_{j - 1} \vee \neg A \vee Q_{j + 1} \vee \cdots \vee Q_{m}, \mid A, s_{1}, s_{2}}
\]

We will need four theorems to implement the resolution tactic, which are all easy to prove:

\begin{itemize}
  \item \mintinline{lean}{resolutionSpecialCase1 {A B C : Prop} : A ∨ B → ¬ A ∨ C → B ∨ C}
  \item \mintinline{lean}{resolutionSpecialCase2 {A B : Prop} : A ∨ B → ¬ A → B}
  \item \mintinline{lean}{resolutionSpecialCase3 {A C : Prop} : A → ¬ A ∨ C → C}
  \item \mintinline{lean}{resolutionSpecialCase4 {A : Prop} : A → ¬ A → False}
\end{itemize}

First, we apply \texttt{pull} to \texttt{pfP} to bring the first occurence of \texttt{A} to the first position. This will result
in a proof \texttt{pfP'} of the clause:

\begin{center}
  $A \vee P_{1} \vee \cdots \vee P_{i - 1} \vee P_{i + 1} \vee \cdots \vee P_{n}$
\end{center}

Next, we do the same at \texttt{pfQ}, bringing the negation of the pivot to the first position. We will obtain a proof \texttt{pfQ'} of
the clause:

\begin{center}
  $\neg A \vee Q_{1} \vee \cdots \vee Q_{j - 1} \vee Q_{j + 1} \vee \cdots \vee Q_{m}$
\end{center}

Now, we apply one of the \texttt{resolutionSpecialCase} theorems to \texttt{pfP'} and \texttt{pfQ'}. We decide which theorem
to apply based on whether or not each clause has other terms apart from the pivot. If both clauses have other terms (i.e. $n > 0$ and
$m > 0$) we apply \texttt{resolutionSpecialCase1}, instantiating \texttt{B} to $P_{1} \vee \cdots \vee P_{i - 1} \vee P_{i + 1} \vee \cdots \vee P_{n}$ and \texttt{C} to $Q_{1} \vee \cdots \vee Q_{j - 1} \vee Q_{j + 1} \vee \cdots \vee Q_{m}$. If the second clause consists of only the negation of the pivot and the first one has other terms, we apply \texttt{resolutionSpecialCase2}. If the first clause consists of only the pivot and the second one has other terms, we apply \texttt{resolutionSpecialCase3}. If both clauses consist of a single element, we apply \texttt{resolutionSpecialCase4}. In this case, the tactic will produce a proof of \texttt{False}, which is the way we represent the empty clause.

Notice that, if we apply \texttt{resolutionSpecialCase1} and the first clause had more than 1 element
apart from the pivot, the term we get is not the required one. Its type will have the
propositions from the first clause parenthesized, i.e.\ it will have the following form:

\begin{center}
  $(P_{1} \vee \cdots \vee P_{n}) \vee Q_{1} \vee \cdots \vee Q_{m}$
\end{center}

We fix this issue with the tactic \texttt{ungroupClausePrefix}, which was mentioned before.

The original rule used by cvc5 has another
boolean parameter, \textit{pol}. If pol is set to true, then the semantics of the rule matches
exactly the tactic we described. If pol is set to false, the rule expects to find the pivot
negated in the first clause and not negated in the second. We have implemented a separate
tactic for this case. The function that implements the core functionality of both tactics is
the same.

The following snippet shows one example of usage of each tactic. \texttt{R1} corresponds to
the rule with pol set to true, and \texttt{R2} corresponds to the rule with pol set to false.
The last parameter of the tactic is a list with two elements, corresponding to the indices
of the last propositions in each clause.

\begin{minted}{lean}
  theorem res1 : A ∨ B ∨ C ∨ D → E ∨ ¬ B → A ∨ (C ∨ D) ∨ E := by
    intros h₁ h₂
    R1 h₁, h₂, B, [2, 1]

  theorem res2 : ¬ A → B ∨ A ∨ C → B ∨ C := by
    intros h₁ h₂
    R2 h₁, h₂, A, [0, 2]
\end{minted}

\subsection*{Removing duplicates}

Given a proof \texttt{pf} of a clause and the index \texttt{s} of the last proposition in the clause,
the \textit{factor} tactic produces a proof of the same clause, with all duplicated literals of it removed. The precise statement of this rule is the following:

\[
  \infer[]{removeDuplicates(P_{1} \vee \cdots \vee P_{n})}{P_{1} \vee \cdots \vee P_{n} \mid s}
\]

We will need two theorems to implement it:

\begin{itemize}
  \item \mintinline{lean}{dupOr {A B : Prop} : A ∨ A ∨ B → A ∨ B}
  \item \mintinline{lean}{dupOr' {A : Prop} : A ∨ A → A}
\end{itemize}

Also, in order to implement factor, we employ the tactic \texttt{pullToMiddle},
a generalization of \texttt{pull}. It allows its user to move a term in position $j$ in some clause to any position
$i < j$. The implementation of \texttt{pullToMiddle} is similar to the one for \texttt{pull}, except
that instead of grouping and ungrouping prefixes of the clause, it considers intervals in the middle of the clause.

The idea we employed to develop this tactic is the following: for each literal in the clause, we
check every other literal to see if it is equal to the current one. If it is, we obtain a new
clause with the literals that are equal adjacent to each other. We then apply either \texttt{dupOr}
or \texttt{dupOr'} (depending whether they are on the last positions of the clause), together
with the appropriate congruence lemmas (like we did for grouping clauses). We repeat this process
until there are no more duplicates in the clause.

The pseudocode in Figure~\ref{factorCore} shows our implementation of this idea.
First, in line 1, we obtain the clause corresponding to \texttt{pf} using \texttt{inferType},
in the same way we did in the implementation of \texttt{groupClausePrefix}. Then, in lines 2
to 15 iterate through each term in the clause.
We use the function $GetLength$ to obtain the current length of the clause. We need to
provide the index of the last proposition to this function, otherwise the length of the clause
will not be well defined.
The goal of each iteration is to remove all other elements of the clause that are equal to
the $i$-th one (the current one). For that, we do another nested loop in lines 5 to 14,
iterating through all indices $j$ such that $j > i$. Next, in line 6, we check whether
the propositions at positions $i$ and $j$ are equal (we use the notation $clause_{i}$ to
indicate the proposition at position $i$). This comparison is syntactic, that is, we
only consider equal propositions that have \textit{exactly} the same representation
as \texttt{Expr}. This is the intended behaviour of this rule. It is designed to
remove duplications introduced by applications of resolution (which will be syntactically equal).
If the propositions are different, we simply increment $j$. Otherwise,
we bring the $j$-th proposition to position $i + 1$ using the tactic \texttt{pullToMiddle}.
This will allow us to use one of the \texttt{dupOr} theorems. If $i + 1$ is the last
position in the clause, we apply \texttt{dupOr'}, otherwise we apply \texttt{dupOr}.
Notice that, since there are potentially other terms to the left of the $i$-th one,
we need to apply \texttt{congOrLeft} composed with \texttt{dupOr}, in the same way we
did for \texttt{groupClausePrefix}. The function $ApplyDupOr$ performs these checks
and applies the correct version of \texttt{dupOr}, producing a new proof term for the clause
with the duplicate erased. Since the number of elements to the left
of the last one necessarily decreased by one, we have to decrement $s$,
which is done in line 9. Finally, we update $clause$
according to our modifications by extracting again the type of $pf$.

Notice that the outer loop maintains following invariant: during the $i$-th iteration,
all propositions on the $i$-th prefix are distinct. Therefore, it is not necessary to
check propositions with index lesser than $i$.

\begin{figure}[t]
\begin{algorithmic}[1]
\Function{FactorCore}{$pf$, $s$}
  \State $clause \gets $ \Call{InferType}{$pf$}
  \For{$i \gets 1 $ \textbf{ up to } \Call{GetLength}{$clause$, $s$}}
    \State $j \gets i + 1$
    \While{$j < $ \Call{GetLength}{$clause$, $s$}}
      \If{$clause_{i} = clause_{j}$}
        \State $pf \gets $ \Call{PullToMiddleCore}{$pf$, $i + 1$, $j$, $s$}
        \State $pf \gets $ \Call{ApplyDupOr}{$pf$, $i$, $i + 1$}
        \State $s \gets s - 1$
        \State $clause \gets $ \Call{InferType}{$pf$}
      \Else
        \State $j \gets j + 1$
      \EndIf
    \EndWhile
  \EndFor
  \State \Return $pf$
\EndFunction
\end{algorithmic}\label{factorCore}
\end{figure}

The following snippet shows an application of this tactic:

\begin{minted}{lean}
  theorem factor1 :
      A ∨ B ∨ (E ∨ F) ∨ B ∨ A ∨ (E ∨ F) → A ∨ B ∨ (E ∨ F) :=
    by intro h
       factor h, 5
\end{minted}

Here, the tactic was used to eliminate the second occurrences of \texttt{A}, \texttt{B} and
\texttt{(E ∨ F)} at the hypothesis \texttt{h}.

\section{Linear Arithmetic Reasoning}

\subsection*{Summing Lists of Inequalities\\Rule Statement:}
\[
  \infer[]{\sum_{i = 1}^{n} a_{i} \bowtie^{*} \sum_{i = 1}^{n} b_{i}}{\bigwedge_{i = 1}^{n} a_{i} \bowtie_{i} b_{i}}
\]

Given a list of $n$ proofs of statements following one of the patterns $a_{i} < b_{i}$, $a_{i} \le b_{i}$ or $a_{i} = b_{i}$, this
tactic produces a proof of the inequality $\sum_{i = 1}^{n} a_{i} \bowtie^{*} \sum_{i = 1}^{n} b_{i}$, where $\bowtie^{*}$ is $<$ if
all terms in the premisses are strict inequalities and $\le$ otherwise.

Each one of the variables $a_{i}$ and $b_{i}$ might be represented inside the SMT solver
with either the \textit{Int} or the \textit{Real} sort. While each pair $a_{i}$ and $b_{i}$ always share the same sort, it is possible that some of these pairs have a sort different from the rest. Before implementing a
tactic corresponding to this rule, we have to choose types in Lean to match
these two sorts. We have based our decision on the types used by mathlib, as
individuals formalizing new mathematical statements in this library are one of the main potential users
of a Lean hammer. Integers are represented in mathlib exclusively with the built-in type \textit{Int}, so we will use it to represent the Int sort.
For the Real sort, we could either represent it using the type \texttt{Real}, which was defined in mathlib
to represent real numbers, or the type \texttt{Rat}, which was defined in the package \textit{std4}\footnote{std4 is Lean's standard library. It can be accessed at: \url{https://github.com/leanprover/std4}} to
represent rational numbers and is used by mathlib to formalize a variety of theorems
that involve this type of number.
It is not a problem to use rational numbers to represent the Real sort since the rules employed by cvc5 in the theory of linear arithmetic
do not explore any particular property that is enjoyed by real numbers and not by
rational numbers.
We decided to employ the type \texttt{Rat}, as its definition is much simpler and easier
to manage in comparison to the one for \texttt{Real}.

The implementation of this tactic will require 9 variations of the following theorem:

\begin{minted}{lean}
  sumBounds {α : Type} [LinearOrderedRing α] {a b c d : α} :
    a < b → c < d → a + c < b + d
\end{minted}

Each variation will correspond to one combination of the relation symbols in the hypothesis, where
they can be either $<$, $\le$ or $=$. The relation symbol in the
conclusion is adapted accordingly in each theorem. Since the rule accepts mixing of variables
from Int and Real sort, we need a variation of each one of those 9 theorems for each combination
of the types of the variables. Instead of stating all the combinations explicitly, which would
result in a total of 36 theorems and a long branch in the implementation of the core functionality of the tactic,
we stated only one polymorphic version of each, as indicated by the type parameter \texttt{α} in \texttt{sumBounds}.
Obviously, the theorem does not hold
for any \texttt{α} (it cannot even be stated if there is no comparison and addition operators defined over \texttt{α}).
We solve this issue by adding a restriction, stating that \texttt{α} satisfies the axioms of a
\textit{Linear Ordered Ring} (a class of types that contains both \texttt{Int} and \texttt{Rat} defined in mathlib), represented by the expression \texttt{[LinearOrderedRing α]} in the theorem. With this restriction we
could find a proof for each theorem.

\begin{figure}[t]
\begin{minted}[linenos]{lean}
  def combineBounds (pf₁ pf₂ : Expr) : MetaM Expr := do
    let t₁ ← inferType pf₁
    let t₂ ← inferType pf₂
    let rel₁ ← getRel t₁
    let rel₂ ← getRel t₂
    let tp₁ ← getOperandType t₁
    let tp₂ ← getOperandType t₂
    let (pf₁, pf₂) ← castHypothesis pf₁ pf₂ rel₁ rel₂ tp₁ tp₂
    let thmName : Name :=
      match rel₁, rel₂ with
      | `LT.lt , `LT.lt => `sumBounds₁
      | `LT.lt , `LE.le => `sumBounds₂
      | _      , _      => panic! "[sumBounds]: invalid relation"
    mkAppM thmName #[pf₂, pf₁]

  def sumBoundsCore (acc : Expr) (pfs : List Expr) : MetaM Expr :=
    match pfs with
    | [] => return acc
    | pf :: pfs' => do
      let acc' ← combineBounds acc pf'
      sumBoundsCore acc' pfs'
\end{minted}
\caption{Implementation of the SumBounds tactic}
\end{figure}

To implement this tactic, we apply the theorems for summing bounds in each element
of the list of proofs received, accumulating the results. We start the accumulator
with the last

After proving the necessary theorems, we can implement the tactic in the following manner:
we process the proofs received as arguments in reverse order, maintaining a variable \texttt{pf}
that accumulates the combined result of all the proofs processed so far, which is initialized
with the last proof in our list. For each proof we go through, we obtain the statement it corresponds to
with \texttt{inferType} and analyze its relation symbol and the relation symbol associated with the
current type of \texttt{pf}. With these two informations we decide which of the variants of the theorem
\texttt{sumBounds} we can apply in this two proofs. Also, we need to check what are the types of the
variables in these proofs to properly instantiate \texttt{α}. If one of them is \texttt{Rat} and the
other is \texttt{Int}, we also have to cast the inequality between integers into
an inequality between the same terms, casted to rationals, since the \texttt{sumBounds} theorems expect all variables to have the same type. This is done using one of the
following three theorems, depending on the relation symbol:

\begin{itemize}
  \item \mintinline{lean}{Int.castEQ : ∀ {a b : Int}, a < b → Rat.ofInt a < Rat.ofInt b}
  \item \mintinline{lean}{Int.castLE : ∀ {a b : Int}, a ≤ b → Rat.ofInt a ≤ Rat.ofInt b}
  \item \mintinline{lean}{Int.castLT : ∀ {a b : Int}, a = b → Rat.ofInt a = Rat.ofInt b}
\end{itemize}

where \texttt{Rat.ofInt} is the standard function to cast an integer into a rational.

\subsection*{MulPosNeg\\Rule Statement:}
% \[
%   \infer[]{}{}
% \]

% \subsection{TightBounds}


% TODO: put this in an appendix
\begin{table}[]\label{tab:rules}
\centering
\begin{tabular}{ l l l }
\toprule
Name        & Statement & Implementation \\ \midrule
NotImplies1 & \texttt{Not (p -> q) -> p}      & theorem        \\ \midrule
NotImplies2 & blah      & theorem        \\ \midrule
EquivElim1  & blah      & theorem        \\ \midrule
EquivElim2  & blah      & theorem        \\ \midrule
NotEquivElim1  & blah      & theorem        \\ \midrule
NotEquivElim2  & blah      & theorem        \\ \midrule
ImpliesElim & blah      & theorem        \\ \bottomrule
\end{tabular}
\caption{Rules from cvc5's proof calculus}
\end{table}


% proof of $P_{i - 1} \vee P_{i} \vee \cdots \vee P_{n} \rightarrow (P_{i - 1} \vee P_{i}) \vee \cdots \vee P_{n}$, together with \texttt{congOrLeft} to surpass
% the propositions to the left of $P_{i - 1}$. We need one application of \texttt{congOrLeft}
% for each proposition to the left of $P_{i - 1}$. For instance, if:
% \begin{itemize}
%   \item \textbf{i = 2} our term will be \texttt{orAssoc}
%   \item \textbf{i = 3} our term will be \texttt{congOrLeft orAssoc}
%   \item \textbf{i = 4} our term will be \texttt{congOrLeft (congOrLeft orAssoc)}
% \end{itemize}


% The idea is to use \texttt{congOrLeft} to surpass the propositions to the left of $P_{i - 1}$ together with \texttt{orAssoc} instantiating \texttt{A} to $P_{i - 1}$, \texttt{B} to $P_{i}$ and \texttt{C} to $P_{i + 1} \vee \cdots \vee
% P_{n}$, which will yield a proof of $P_{i - 1} \vee P_{i} \vee \cdots \vee P_{n} \rightarrow (P_{i - 1} \vee P_{i}) \vee \cdots \vee P_{n}$.


% We cannot directly apply it to $h$ since there are potentially other terms to the left of $P_{i - 1}$.
% To solve this problem, we use the theorem \texttt{congOrLeft} setting \texttt{hyp} to the application of \texttt{orAssoc} that we just described.
% Since the first $i - 2$ propositions are not grouped, we cannot instantiate \texttt{C} to $P_{i - 1} \vee \cdots \vee P_{i - 2}$.
% Instead, we set \texttt{C} to $P_{i - 2}$, which will yield a proof \texttt{h₂} of $P_{i - 2} \vee P_{i - 1} \vee P_{i} \vee \cdots \vee P_{n} \rightarrow P_{i - 2} \vee (P_{i - 1} \vee P_{i}) \vee \cdots \vee P_{n}$. Now we repeat the process applying \texttt{congOrLeft} with \texttt{C} instantiated to $P_{i - 3}$ (if it exists) and \texttt{hyp} to \texttt{h₂}. We keep going until we set \texttt{C} to $P_{1}$. We will then obtain a term of the form \texttt{congOrLeft (congOrLeft \ldots (congOrLeft orAssoc)\ldots)} (with $i - 2$ applications of \texttt{congOrLeft}), which, when applied to \texttt{h}, will prove $P_{1} \vee \cdots \vee (P_{i - 1} \vee P_{i}) \vee \cdots \vee P_{n}$.


% One detail that has to be considered while implementing this tactic is that the index of the last proposition
% may change when we pull a term. For instance, the index of the last proposition of the clause
% $A \vee B \vee (C \vee D) \vee E$ is $4$. If we pull $E$ in this clause, we will get
% $E \vee A \vee B \vee (C \vee D)$, which has $3$ as this index. If this happens,
% the new index will always be the original length of the clause minus the length
% of the second to last proposition in the current clause (since it will become the last proposition).
% While executing the procedure we described in the last paragraph, we need to always keep track of the
% current index of the last proposition, as the tactic pull requires it as an argument.

% \begin{figure}[t]
% % \textbf{Input:} $\psi$, a PL formula\\
% % \textbf{Output:} \textit{true} or \textit{false}, depending whether $\psi$ is satisfiable
% \begin{algorithmic}[1]
% \Function{PermutateClauseCore}{$pf$, $perm$, $s$}
%   \State $clause \gets $ \Call{InferType}{$pf$}
%   \State $lenClause \gets $ \Call{getLength}{$clause$}
%   \For{$i \gets n \Downto 1$}
%   \If{$perm_{i} = s$}
%     \State $currentClause \gets$ \Call{InferType}{pf}
%     \State $lenSecondLast \gets $ \Call{GetLength}{$currentClause_{s - 1}$}
%     \State $pf \gets$ \Call{PullCore}{$pf$, $perm_{i}$, $s$}
%     \State $s \gets lenSecondLast - lenClause$
%   \Else
%     \State $pf \gets$ \Call{PullCore}{$pf$, $perm_{i}$, $s$}
%   \EndIf
%   \EndFor
%   \State \Return~$pf$
% \EndFunction
% \end{algorithmic}
% \caption{Implementation of the tactic PermutateClause}~\label{permClause}
% \end{figure}

	  % \section{Certified vs Certifying}
    %   \label{sec:certifiedVsCertifying}
	  % \section{Classical vs Intuitionist}
	  % \section{Tactics}
	  % \section{The Complete Architecture}
	  % \section{Skipping the Parser}
	  % maybe think of other optimizations and make a section about this
	\chapter{Evaluation}
	\chapter{Future Work}

		%% Referências
		\bibliographystyle{plain}
		\bibliography{referencias}

\end{document}
